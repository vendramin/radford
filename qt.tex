\section{Quasitriangular Hopf algebras}

%\subsection{Braided vector spaces}

\begin{definition}
\index{braided vector space}
A \underline{braided vector space} is a pair $\left(V,c\right)$, where
$V$ is a vector space and $c\in\mathbf{GL}(V\otimes V)$ is a solution
of the braid equation:
\[
(c\otimes\id)(\id\otimes c)(c\otimes\id)=(\id\otimes c)(c\otimes\id)(\id\otimes c).
\]
\end{definition}

\begin{example}
Let $V$ be any vector space. Let $\tau:V\to V$ be the linear map
defined by $\tau(x\otimes y)=y\otimes x$ for all $x,y\in V$. The
pair $(V,\tau)$ is a braided vector space. 
\end{example}

\begin{example}
Let $G$ be a finite group and $V=\mathbb{K}G$ be the vector space
with basis $\{g\mid g\in G\}$. Define $c(g\otimes h)=ghg^{-1}\otimes g$.
Then $(V,c)$ is a braided vector space.
\end{example}

\begin{exercise}
Let $(V,c)$ be a braided vector space. Prove that the pairs $(V,\lambda c)$,
$(V,c^{-1})$ and $(V,\tau\circ c\circ\tau)$ are also braided vector
spaces, where $\lambda$ is any non-zero scalar.
\end{exercise}

%\subsection{Quasitriangular Hopf algebras}

Let $A$ be an algebra (over the field $\mathbb{K}$) and suppose
that $R=\sum_{i=1}^{n}a_{i}\otimes b_{i}\in A\otimes A$ is invertible.
Define 
\[
R_{12}=\sum_{i=1}^{n}a_{i}\otimes b_{i}\otimes1,\quad R_{13}=\sum_{i=1}^{n}a_{i}\otimes1\otimes b_{i},\quad R_{23}=\sum_{i=1}^{n}1\otimes a_{i}\otimes b_{i}.
\]

\begin{definition}
\index{Hopf algebra!quasitriangular}
\index{quasitriangular Hopf algebra}
A \underline{quasitriangular} Hopf algebra is a pair $(H,R)$, where
$H$ is a Hopf algebra and $R=\sum_{i}a_{i}\otimes b_{i}\in H\otimes H$
is an invertible elemmaent such that the following conditions are satisfied:
\begin{align}
(\Delta\otimes\textrm{id})(R) & =R_{13}R_{23},\label{QT:2}\\
(\textrm{id}\otimes\Delta)(R) & =R_{13}R_{12},\label{QT:3}\\
\tau\Delta(h)R & =R\Delta(h)\label{QT:1}
\end{align}
for all $h\in H$.
\end{definition}

\begin{remark}
Using Sweedler notation, Equations \eqref{QT:1}--\eqref{QT:3} can
be written as:
\begin{align*}
\sum h_{2}a_{i}\otimes h_{1}b_{i} & =\sum a_{i}h_{1}\otimes b_{i}h_{2},\\
\sum a_{i,1}\otimes a_{i,2}\otimes b_{i} & =\sum a_{i}\otimes a_{j}\otimes b_{i}b_{j},\\
\sum a_{i}\otimes b_{i,1}\otimes b_{i,2} & =\sum a_{i}a_{j}\otimes b_{j}\otimes b_{i}.
\end{align*}
\end{remark}

\begin{example}
\index{Hopf algebra!cocommutative}
\index{cocommutative Hopf algebra}
Let $H$ be a \underline{cocommutative} Hopf algebra, i.e.,
$\tau\circ\Delta=\Delta$.  The pair $(H,R)$, where $R=1\otimes1$, is a
quasitriangular Hopf algebra.
\end{example}

\begin{example}
Let $H=\mathbb{C}\mathbb{Z}_{2}$ be the group algebra of $\langle
g\rangle\simeq\mathbb{Z}_{2}$ with the usual Hopf algebra structure. Let 
\[
R=\frac{1}{2}(1\otimes1+1\otimes g+g\otimes1-g\otimes g).
\]
Then $(H,R)$ is a quasitriangular Hopf algebra.
\end{example}

\begin{example}
Recall that the Sweedler 4-dimensional algebra $H$ is the algebra (say over
$\mathbb{C}$) generated by $x,y$ with relations $x^{2}=1$, $y^{2}=0$ and
$xy+yx=0$.  The Hopf algebra structure is given by $\Delta(x)=x\otimes x$,
$\Delta(y)=1\otimes y+y\otimes x$, $\varepsilon(x)=1$, $\varepsilon(y)=0$,
$S(x)=x$ and $S(y)=xy$.  A linear basis for $H$ is $\{1,x,y,xy\}$. Let
\[
R_{\lambda}=\frac{1}{2}(1\otimes1+1\otimes x+x\otimes1-x\otimes x)+\frac{\lambda}{2}(y\otimes y+y\otimes xy+xy\otimes xy-xy\otimes y)
\]
where $\lambda$ is any scalar. Then $(H,R_{\lambda})$ is a quasitriangular
Hopf algebra. Observe that $\tau(R_{\lambda})=R_{\lambda}^{-1}$.
\end{example}

\begin{definition}
\index{Hopf algebra!triangular}
\index{triangular Hopf algebra}
A \underline{triangular} Hopf algebra is a quasitriangular Hopf algebra
$(H,R)$ such that $\tau(R)=R^{-1}$. 
\end{definition}

\begin{exercise}
Let $(H,R)$ be a quasitriangular Hopf algebra with comultiplication $\Delta$
and bijective antipode $S$. Prove that $(H^{\mathrm{cop}},\tau(R))$ is also a
quasitriangular Hopf algebra. (Recall that $H^{\mathrm{cop}}$ is the Hopf
algebra structure over $H$ with comultiplication
$\Delta^{\mathrm{op}}=\tau\circ\Delta$ and antipode $S^{-1}$.)
\end{exercise}

\begin{proposition}
Let $(H,R)$ be a quasitriangular Hopf algebra with bijective antipode.
Then
\begin{gather}
(\varepsilon\otimes\id)(R)=(\id\otimes\varepsilon)(R)=1,\label{eq:qt_epsilon}\\
(S\otimes\id)(R)=(\id\otimes S^{-1})(R)=R^{-1},\label{eq:qt_sx1}\\
(S\otimes S)(R)=R.\label{eq:qt_sxs}
\end{gather}
\end{proposition}

\begin{proof}
We first prove \eqref{eq:qt_epsilon}. Apply $\varepsilon\otimes\textrm{id}\otimes\textrm{id}$
to $(\Delta\otimes\textrm{id})(R)=R_{13}R_{23}$ to obtain 
\[
R=\sum a_{i}\otimes b_{i}=\sum(\varepsilon\otimes\textrm{id})\Delta(a_{i})\otimes b_{i}=\sum\varepsilon(a_{i})a_{j}\otimes b_{i}b_{j}=(\varepsilon\otimes\textrm{id})(R)R.
\]
and the claim follows since $R$ is invertible. The other claim in
\eqref{eq:qt_epsilon} is similar: one needs to apply
$\textrm{id}\otimes\textrm{id}\otimes\varepsilon$ to
$(\textrm{id}\otimes\Delta)(R)=R_{13}R_{12}$.

Now we prove \eqref{eq:qt_sx1}.  Apply $(m\otimes\id)(S\otimes\id\otimes\id)$
to $(\Delta\otimes\id)(R)=R_{13}R_{23}$ to obtain 
\[
(m\otimes\id)(S\otimes\id\otimes\id)(\Delta\otimes\id)(R)=(\eta\varepsilon\otimes\id)(R)=(\varepsilon\otimes\id)(R)=1\otimes1.
\]
On the other hand 
\[
1\otimes 1=m(S\otimes\id)(R_{13}R_{23})=\sum S(a_{i})a_{j}\otimes b_{i}b_{j}=(S\otimes\id)(R)R.
\]
Hence $(S\otimes\id)(R)=R^{-1}$ since $R$ is invertible.
To prove $(\textrm{id}\otimes S^{-1})(R)=R^{-1}$ notice that $(H^{\textrm{cop}},\tau(R))$
is a quasitriangular Hopf algebra.

Finally, \eqref{eq:qt_sxs} follows from \eqref{eq:qt_sx1} since
\[
(S\otimes S)(R)=(\textrm{id}\otimes S)(S\otimes\textrm{id})(R)=(\textrm{id}\otimes S)(R^{-1})=R.
\]
This completes the proof.
\end{proof}

\begin{proposition}
\label{proposition:R-matrix}
Let $(H,R)$ be a quasitriangular Hopf algebra
with bijective antipode. Then 
\begin{equation}
R_{12}R_{13}R_{23}=R_{23}R_{13}R_{12}.\label{eq:121323=231312}
\end{equation}
\end{proposition}

\begin{proof}
Using \eqref{QT:1} and \eqref{QT:2} we obtain 
\begin{align*}
R_{12}R_{13}R_{23} & =R_{12}(\Delta\otimes\textrm{id})(R)=(\Delta^{\mathrm{op}}\otimes\id)(R)R_{12}\\
 & =(\tau\otimes\id)(\Delta\otimes\id)(R)R_{12}=(\tau\otimes\id)(R_{13}R_{23})R_{12}=R_{23}R_{13}R_{12}.
\end{align*}
This proves the claim.
\end{proof}

\begin{exercise}
Write Equations \eqref{eq:qt_epsilon}, \eqref{eq:qt_sx1}, \eqref{eq:qt_sxs}
and \eqref{eq:121323=231312} using Sweedler notation.
\end{exercise}

\label{paragraph:QT_braiding}
Let $(H,R)$ be a quasitriangular Hopf algebra, and let $V$ and $W$ be
two left $H$-modules. Assume that $R=\sum a_{i}\otimes b_{i}$ and define the map
\begin{align*}
R_{V,W}:V\otimes W&\to W\otimes V\\
v\otimes w&\mapsto \tau_{V,W}\left(R\cdot (v\otimes w)\right)=\sum b_{i}\cdot w\otimes a_{i}\cdot v.
\end{align*}

The map $R_{V,W}$ is invertible and 
\[
\left(R_{V,W}\right)^{-1}(w\otimes v)=R^{-1}\cdot(v\otimes w).
\]

\begin{lemma}
The map $R_{V,W}$ is an isomorphism of $H$-modules.
\end{lemma}

\begin{proof}
First compute 
\begin{align*}
R_{V,W}\left(h\cdot(v\otimes w)\right) & =\sum\tau_{V,W}\left(R(h_{1}\cdot v\otimes h_{2}\cdot w)\right)\\
 & =\sum\tau_{V,W}\left((a_{_{i}}h_{1})\cdot v\otimes(b_{i}h_{2})\cdot w\right)\\
 & =\sum(b_{i}h_{2})\cdot w\otimes(a_{i}h_{1})\cdot v.
\end{align*}
On the other hand,
\begin{eqnarray*}
h\cdot R_{V,W}(v\otimes w) & = & \sum h_{1}\cdot(b_{i}\cdot w)\otimes h_{2}\cdot(a_{i}\cdot v)\\
 & = & \sum(h_{1}b_{i})\cdot w\otimes(h_{2}a_{i})\cdot v.
\end{eqnarray*}
Apply \eqref{QT:1} to $h$ and the claim follows.
\end{proof}

\begin{proposition}
\label{proposition:braid_equation}
Let $(H,R)$ be a quasitriangular Hopf algebra, and let $V$ and $W$ be two left
$H$-modules. Then 
\begin{align} 
(R_{V,W}\otimes\id_{U}) & (\id_{V}\otimes
R_{U,W})(R_{U,V}\mathrm{\otimes id}_{W})\nonumber \\ & =(\id_{W}\otimes
R_{U,V})(R_{U,W}\otimes\id_{V})(\id_{U}\otimes
R_{V,W}).\label{eq:braid_equation} 
\end{align}
\end{proposition}

\begin{proof}
A direct computation shows that
\begin{align*}
(R_{V,W}\otimes\id_{U})(\id_{V}\otimes R_{U,W}) & (R_{U,V}\mathrm{\otimes id}_{W})(u\otimes v\otimes w)\\
 & =\sum(b_{k}b_{j})\cdot w\otimes(a_{k}b_{i})\cdot v\otimes(a_{j}a_{i})\cdot u.
\end{align*}
On the other hand,
\begin{align*}
(\id_{W}\otimes R_{U,V})(R_{U,W}\otimes\id_{V}) & (\id_{U}\otimes R_{V,W})(u\otimes v\otimes w)\\
 & =\sum(b_{j}b_{i})\cdot w\otimes(b_{k}a_{i})\cdot v\otimes(a_{k}a_{j})\cdot u
\end{align*}
and hence the claim follows from propositionosition \ref{proposition:R-matrix}.\end{proof}

\begin{exercise}
\label{exercise:QT_hexagons}
Let $(H,R)$ be a quasitriangular Hopf algebra, and let $U$, $V$ and $W$
be three left $H$-modules. Prove that 
\begin{align}
R_{U\otimes V,W} & =(R_{U,W}\otimes\id_{V})(\id_{U}\otimes R_{V,W}),\label{eq:(Rx1)(1xR)}\\
R_{U,V\otimes W} & =(\id_{V}\otimes R_{U,W})(R_{U,V}\otimes\id_{W}).\label{eq:(1xR)(Rx1)}
\end{align}
\end{exercise}

Setting $U=V=W$ in proposition \ref{proposition:braid_equation} we conclude that
$R_{V,V}$ is a solution of the braid equation for any left $H$-module $V$. 

\begin{definition}
\index{Hopf algebra!almost cocommutative}
\index{almost cocommutative Hopf algebra}
A Hopf algebra $H$ is called \underline{almost cocommutative} if there exists
an invertible elemmaent $R\in H\otimes H$ such that $\tau\Delta(h)R=R\Delta(h)$
for all $h\in H$.
\end{definition}

\begin{proposition}
Let $(H,R)$ be an almost cocommutative Hopf algebra. Then $S^2$ is an inner
automorphism of $H$. More precisely, assume that $R=\sum a_i\otimes b_i$, and
let $u=\sum (Sb_i)a_i$. Then $u$ is invertible in $H$ and
\[
S^2h=uhu^{-1}=(Su)^{-1}h(Su)
\]
for all $h\in H$.
\end{proposition}

\begin{proof}
First we prove that $uh=(S^2h)u$ for all $h\in H$. Since $H$ is almost cocommutative, 
$(R\otimes1)(h_1\otimes h_2\otimes h_3)=(h_2\otimes h_1\otimes h_3)(R\otimes1)$, i.e.,
\[
\sum a_ih_1\otimes b_ih_2\otimes h_3=\sum h_2a_i\otimes h_1b_1\otimes h_3.
\]
Then
\[
\sum S^2h_3S(b_ih_2)a_ih_1=\sum (S^2h_3)S(h_1b_i)h_2a_i.
\]
Using propositionerties of the antipode and the counit we obtain: 
\[
\sum S^2h_3S(b_ih_2)a_ih_1=\sum S(h_2Sh_3)(Sb_i)a_ih_1= \sum (Sb_i)a_ih=uh.
\]
Similarly, 
\[
\sum (S^2h_3)S(h_1b_i)h_2a_i=\sum S^2h_3(Sb_i)(Sh_1)h_2a_i=\sum (S^2h)(Sb_i)a_i=(S^2h)u
\]
and hence $uh=(S^2h)u$ for all $h\in H$.

Now we prove that $u$ is invertible. Assume that $R^{-1}=\sum c_j\otimes d_j$ and
let $v=\sum S^{-1}(d_j)c_j$. Since $uh=(S^2h)u$, we obtain: 
\begin{align*}
uv=\sum_j u(S^{-1}d_j)c_j&=\sum_j (Sd_j)uc_j\\
&=\sum_{i,j} (Sd_j)(Sb_i)a_ic_j=\sum_{i,j} S(b_id_j)a_ic_j.
\end{align*}
Therefore $uv=1$ since $1\otimes1=RR^{-1}=\sum_{i,j}a_ic_j\otimes b_id_j$.
Using $S^2h=uhu^{-1}$ with $h=v$ we obtain $1=S^2(v)u$ and hence $u$ is
invertible. 

The formula $S^2h=(Su)^{-1}h(Su)$ follows from applying $S$ to
$S^2h=uhu^{-1}$ and replacing $Sh$ by $h$.
\end{proof}

\begin{corollary}
Let $(H,R)$ be an almost cocommutative Hopf algebra. Then the elemmaent $u(Su)$
is central in $H$.
\end{corollary}

\begin{exercise}
\label{exercise:VW=WV}
Let $H$ be an almost cocommutative Hopf algebra. Let $V$ and $W$ be two left
$H$-modules.  Then $V\otimes W\simeq W\otimes V$ as left $H$-modules.
\end{exercise}

%\subsection{Coquasitriangular Hopf algebras}
%
%\begin{definition}
%	A \underline{coquasitriangular} Hopf algebra is a Hopf algebra $H$ together with a
%	convolution-invertible bilinear form
%	$\langle\cdot|\cdot\rangle:H\otimes H\to\mathbb{K}$ such that 
%	\begin{align*}
%		\langle h_1|k_1\rangle k_2h_2 &= h_1k_1\langle h_2|k_2\rangle,\\
%		\langle h|kl\rangle &= \langle h_1|k\rangle\langle h_2|l\rangle,\\
%		\langle hk|l\rangle &= \langle h|l_2\rangle\langle k|l_1\rangle,
%	\end{align*}
%	for all $h,k,l\in H$.
%\end{definition}
%
%\begin{exercise}
%	Let $H$ be a finite-dimensional Hopf algebra. Prove that $H$ is
%	coquasitriangular if and only if $H^*$ is quasitriangular.
%\end{exercise}
%
%\begin{solution}
%	blah
%\end{solution}
%
%\begin{example}
%	Let $H$ be a commutative Hopf algebra. Then $H$ is coquasitriangular with
%	$\langle h|k\rangle=\varepsilon(h)\varepsilon(k)$.
%\end{example}
%
%\begin{example}
%	Let $G$ be a group and $G=\mathbb{K}[G]$ be the group algebra. Then $H$ is coquasitriangular
%	if and only if $G$ is abelian, 
%	\[
%	\langle h|gl\rangle =\langle h|g\rangle\langle h|l\rangle\quad\text{ and }\quad 
%	\langle hg|l\rangle =\langle h|l\rangle \langle g|l\rangle
%	\]
%	for all $g,h,l\in G$.
%\end{example}
%
%\begin{exercise}
%	Let $H$ be a \underline{almost commutative} Hopf algebra, i.e.,  
%	\[
%		\langle h_1|k_1\rangle k_2h_2 = h_1k_1\langle h_2|k_2\rangle
%	\]
%	for all $h,k\in H$. Let $V$ and $W$ be two right $H$-comodules. Prove that
%	\[
%	V\otimes W\simeq W\otimes V
%	\]
%	as right $H$-comodules.
%\end{exercise}
%
