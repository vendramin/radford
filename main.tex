\documentclass[graybox]{svmult}

\usepackage{amssymb}
\usepackage{amstext}
\usepackage{mathtools}
\usepackage{mathptmx}       
\usepackage{helvet}        
\usepackage{courier}      
\usepackage{type1cm}     
\usepackage{listings}                        
\usepackage{makeidx}    
\usepackage{graphicx}   
\usepackage{multicol}        
\usepackage[bottom]{footmisc}

% For mathcal fonts
\usepackage{eucal} 

\input xy
\xyoption{all}

%\makeatletter
%\makeatother

\let\remark\relax
\let\theorem\relax
\let\lemma\relax
\let\definition\relax
\let\proposition\relax
\let\corollary\relax
\let\exercise\relax
\let\example\relax
\let\conjecture\relax
\spnewtheorem{theorem}{\theoremname}{\bfseries}{\itshape}
\renewcommand\thetheorem{\arabic{theorem}}
\spnewtheorem{lemma}[theorem]{\lemmaname}{\bfseries}{\itshape}
\spnewtheorem{definition}[theorem]{\definitionname}{\bfseries}{\itshape}
\spnewtheorem{proposition}[theorem]{\propositionname}{\bfseries}{\itshape}
\spnewtheorem{corollary}[theorem]{\corollaryname}{\bfseries}{\itshape}
\spnewtheorem{exercise}[theorem]{\exercisename}{\bfseries}{\upshape}
\spnewtheorem{example}[theorem]{\examplename}{\bfseries}{\upshape}
\spnewtheorem{examples}[theorem]{\examplesname}{\bfseries}{\upshape}
\spnewtheorem{remark}[theorem]{\remarkname}{}{\upshape}
\spnewtheorem{conjecture}[theorem]{\conjecturename}{\bfseries}{\upshape}
\spnewtheorem{notation}[theorem]{\notationname}{\bfseries}{\upshape}
\spnewtheorem{convention}[theorem]{\conventionname}{\bfseries}{\upshape}

\newcommand{\lmod}[1]{{}_{#1}\mathcal{M}} 
\newcommand{\rmod}[1]{\mathcal{M}_{#1}} 
\newcommand{\lcomod}[1]{\;^{#1}\mathcal{M}} 
\newcommand{\rcomod}[1]{\mathcal{M}^{#1}}
\renewcommand{\I}{\mathbb{I}}
\newcommand{\id}{\mathrm{id}}

\newcommand{\ydG}{\prescript{G}{G}{\mathcal{YD}}}
\newcommand{\ydH}{\prescript{H}{H}{\mathcal{YD}}}

\makeindex

\begin{document}

%\tableofcontents{}

\title*{Radford's theorem}
\author{Leandro Vendramin}

\institute{Leandro Vendramin \at Vrije Universiteit Brussel, \email{Leandro.Vendramin@vub.be}}

\maketitle

\abstract{
This is a minicourse on Radford's bosonization theorem. 
}

\section{Quasitriangular Hopf algebras}

%\subsection{Braided vector spaces}

\begin{definition}
\index{braided vector space}
A \textbf{braided vector space} is a pair $\left(V,c\right)$, where
$V$ is a vector space and $c\in\mathbf{GL}(V\otimes V)$ is a solution
of the braid equation:
\[
(c\otimes\id)(\id\otimes c)(c\otimes\id)=(\id\otimes c)(c\otimes\id)(\id\otimes c).
\]
\end{definition}

\begin{example}
Let $V$ be any vector space. Let $\tau:V\to V$ be the linear map
defined by $\tau(x\otimes y)=y\otimes x$ for all $x,y\in V$. The
pair $(V,\tau)$ is a braided vector space. 
\end{example}

\begin{example}
Let $G$ be a finite group and $V=\mathbb{K}G$ be the vector space
with basis $\{g\mid g\in G\}$. Define $c(g\otimes h)=ghg^{-1}\otimes g$.
Then $(V,c)$ is a braided vector space.
\end{example}

\begin{exercise}
Let $(V,c)$ be a braided vector space. Prove that the pairs $(V,\lambda c)$,
$(V,c^{-1})$ and $(V,\tau\circ c\circ\tau)$ are also braided vector
spaces, where $\lambda$ is any non-zero scalar.
\end{exercise}

%\subsection{Quasitriangular Hopf algebras}

Let $A$ be an algebra (over the field $\mathbb{K}$) and suppose
that $R=\sum_{i=1}^{n}a_{i}\otimes b_{i}\in A\otimes A$ is invertible.
Define 
\[
R_{12}=\sum_{i=1}^{n}a_{i}\otimes b_{i}\otimes1,\quad R_{13}=\sum_{i=1}^{n}a_{i}\otimes1\otimes b_{i},\quad R_{23}=\sum_{i=1}^{n}1\otimes a_{i}\otimes b_{i}.
\]

\begin{definition}
\index{Hopf algebra!quasitriangular}
\index{quasitriangular Hopf algebra}
A \textbf{quasitriangular} Hopf algebra is a pair $(H,R)$, where
$H$ is a Hopf algebra and $R=\sum_{i}a_{i}\otimes b_{i}\in H\otimes H$
is an invertible elemmaent such that the following conditions are satisfied:
\begin{align}
(\Delta\otimes\textrm{id})(R) & =R_{13}R_{23},\label{QT:2}\\
(\textrm{id}\otimes\Delta)(R) & =R_{13}R_{12},\label{QT:3}\\
\tau\Delta(h)R & =R\Delta(h)\label{QT:1}
\end{align}
for all $h\in H$.
\end{definition}

\begin{remark}
Using Sweedler notation, Equations \eqref{QT:1}--\eqref{QT:3} can
be written as:
\begin{align*}
\sum h_{2}a_{i}\otimes h_{1}b_{i} & =\sum a_{i}h_{1}\otimes b_{i}h_{2},\\
\sum a_{i,1}\otimes a_{i,2}\otimes b_{i} & =\sum a_{i}\otimes a_{j}\otimes b_{i}b_{j},\\
\sum a_{i}\otimes b_{i,1}\otimes b_{i,2} & =\sum a_{i}a_{j}\otimes b_{j}\otimes b_{i}.
\end{align*}
\end{remark}

\begin{example}
\index{Hopf algebra!cocommutative}
\index{cocommutative Hopf algebra}
Let $H$ be a \textbf{cocommutative} Hopf algebra, i.e.,
$\tau\circ\Delta=\Delta$.  The pair $(H,R)$, where $R=1\otimes1$, is a
quasitriangular Hopf algebra.
\end{example}

\begin{example}
Let $H=\mathbb{C}\mathbb{Z}_{2}$ be the group algebra of $\langle
g\rangle\simeq\mathbb{Z}_{2}$ with the usual Hopf algebra structure. Let 
\[
R=\frac{1}{2}(1\otimes1+1\otimes g+g\otimes1-g\otimes g).
\]
Then $(H,R)$ is a quasitriangular Hopf algebra.
\end{example}

\begin{example}
Recall that the Sweedler 4-dimensional algebra $H$ is the algebra (say over
$\mathbb{C}$) generated by $x,y$ with relations $x^{2}=1$, $y^{2}=0$ and
$xy+yx=0$.  The Hopf algebra structure is given by $\Delta(x)=x\otimes x$,
$\Delta(y)=1\otimes y+y\otimes x$, $\varepsilon(x)=1$, $\varepsilon(y)=0$,
$S(x)=x$ and $S(y)=xy$.  A linear basis for $H$ is $\{1,x,y,xy\}$. Let
\[
R_{\lambda}=\frac{1}{2}(1\otimes1+1\otimes x+x\otimes1-x\otimes x)+\frac{\lambda}{2}(y\otimes y+y\otimes xy+xy\otimes xy-xy\otimes y)
\]
where $\lambda$ is any scalar. Then $(H,R_{\lambda})$ is a quasitriangular
Hopf algebra. Observe that $\tau(R_{\lambda})=R_{\lambda}^{-1}$.
\end{example}

\begin{definition}
\index{Hopf algebra!triangular}
\index{triangular Hopf algebra}
A \textbf{triangular} Hopf algebra is a quasitriangular Hopf algebra
$(H,R)$ such that $\tau(R)=R^{-1}$. 
\end{definition}

\begin{exercise}
Let $(H,R)$ be a quasitriangular Hopf algebra with comultiplication $\Delta$
and bijective antipode $S$. Prove that $(H^{\mathrm{cop}},\tau(R))$ is also a
quasitriangular Hopf algebra. (Recall that $H^{\mathrm{cop}}$ is the Hopf
algebra structure over $H$ with comultiplication
$\Delta^{\mathrm{op}}=\tau\circ\Delta$ and antipode $S^{-1}$.)
\end{exercise}

\begin{proposition}
Let $(H,R)$ be a quasitriangular Hopf algebra with bijective antipode.
Then
\begin{gather}
(\varepsilon\otimes\id)(R)=(\id\otimes\varepsilon)(R)=1,\label{eq:qt_epsilon}\\
(S\otimes\id)(R)=(\id\otimes S^{-1})(R)=R^{-1},\label{eq:qt_sx1}\\
(S\otimes S)(R)=R.\label{eq:qt_sxs}
\end{gather}
\end{proposition}

\begin{proof}
We first prove \eqref{eq:qt_epsilon}. Apply $\varepsilon\otimes\textrm{id}\otimes\textrm{id}$
to $(\Delta\otimes\textrm{id})(R)=R_{13}R_{23}$ to obtain 
\[
R=\sum a_{i}\otimes b_{i}=\sum(\varepsilon\otimes\textrm{id})\Delta(a_{i})\otimes b_{i}=\sum\varepsilon(a_{i})a_{j}\otimes b_{i}b_{j}=(\varepsilon\otimes\textrm{id})(R)R.
\]
and the claim follows since $R$ is invertible. The other claim in
\eqref{eq:qt_epsilon} is similar: one needs to apply
$\textrm{id}\otimes\textrm{id}\otimes\varepsilon$ to
$(\textrm{id}\otimes\Delta)(R)=R_{13}R_{12}$.

Now we prove \eqref{eq:qt_sx1}.  Apply $(m\otimes\id)(S\otimes\id\otimes\id)$
to $(\Delta\otimes\id)(R)=R_{13}R_{23}$ to obtain 
\[
(m\otimes\id)(S\otimes\id\otimes\id)(\Delta\otimes\id)(R)=(\eta\varepsilon\otimes\id)(R)=(\varepsilon\otimes\id)(R)=1\otimes1.
\]
On the other hand 
\[
1\otimes 1=m(S\otimes\id)(R_{13}R_{23})=\sum S(a_{i})a_{j}\otimes b_{i}b_{j}=(S\otimes\id)(R)R.
\]
Hence $(S\otimes\id)(R)=R^{-1}$ since $R$ is invertible.
To prove $(\textrm{id}\otimes S^{-1})(R)=R^{-1}$ notice that $(H^{\textrm{cop}},\tau(R))$
is a quasitriangular Hopf algebra.

Finally, \eqref{eq:qt_sxs} follows from \eqref{eq:qt_sx1} since
\[
(S\otimes S)(R)=(\textrm{id}\otimes S)(S\otimes\textrm{id})(R)=(\textrm{id}\otimes S)(R^{-1})=R.
\]
This completes the proof.
\end{proof}

\begin{proposition}
\label{proposition:R-matrix}
Let $(H,R)$ be a quasitriangular Hopf algebra
with bijective antipode. Then 
\begin{equation}
R_{12}R_{13}R_{23}=R_{23}R_{13}R_{12}.\label{eq:121323=231312}
\end{equation}
\end{proposition}

\begin{proof}
Using \eqref{QT:1} and \eqref{QT:2} we obtain 
\begin{align*}
R_{12}R_{13}R_{23} & =R_{12}(\Delta\otimes\textrm{id})(R)=(\Delta^{\mathrm{op}}\otimes\id)(R)R_{12}\\
 & =(\tau\otimes\id)(\Delta\otimes\id)(R)R_{12}=(\tau\otimes\id)(R_{13}R_{23})R_{12}=R_{23}R_{13}R_{12}.
\end{align*}
This proves the claim.
\end{proof}

\begin{exercise}
Write Equations \eqref{eq:qt_epsilon}, \eqref{eq:qt_sx1}, \eqref{eq:qt_sxs}
and \eqref{eq:121323=231312} using Sweedler notation.
\end{exercise}

\label{paragraph:QT_braiding}
Let $(H,R)$ be a quasitriangular Hopf algebra, and let $V$ and $W$ be
two left $H$-modules. Assume that $R=\sum a_{i}\otimes b_{i}$ and define the map
\begin{align*}
R_{V,W}:V\otimes W&\to W\otimes V\\
v\otimes w&\mapsto \tau_{V,W}\left(R\cdot (v\otimes w)\right)=\sum b_{i}\cdot w\otimes a_{i}\cdot v.
\end{align*}

The map $R_{V,W}$ is invertible and 
\[
\left(R_{V,W}\right)^{-1}(w\otimes v)=R^{-1}\cdot(v\otimes w).
\]

\begin{lemma}
The map $R_{V,W}$ is an isomorphism of $H$-modules.
\end{lemma}

\begin{proof}
First compute 
\begin{align*}
R_{V,W}\left(h\cdot(v\otimes w)\right) & =\sum\tau_{V,W}\left(R(h_{1}\cdot v\otimes h_{2}\cdot w)\right)\\
 & =\sum\tau_{V,W}\left((a_{_{i}}h_{1})\cdot v\otimes(b_{i}h_{2})\cdot w\right)\\
 & =\sum(b_{i}h_{2})\cdot w\otimes(a_{i}h_{1})\cdot v.
\end{align*}
On the other hand,
\begin{eqnarray*}
h\cdot R_{V,W}(v\otimes w) & = & \sum h_{1}\cdot(b_{i}\cdot w)\otimes h_{2}\cdot(a_{i}\cdot v)\\
 & = & \sum(h_{1}b_{i})\cdot w\otimes(h_{2}a_{i})\cdot v.
\end{eqnarray*}
Apply \eqref{QT:1} to $h$ and the claim follows.
\end{proof}

\begin{proposition}
\label{proposition:braid_equation}
Let $(H,R)$ be a quasitriangular Hopf algebra, and let $V$ and $W$ be two left
$H$-modules. Then 
\begin{align} 
(R_{V,W}\otimes\id_{U}) & (\id_{V}\otimes
R_{U,W})(R_{U,V}\mathrm{\otimes id}_{W})\nonumber \\ & =(\id_{W}\otimes
R_{U,V})(R_{U,W}\otimes\id_{V})(\id_{U}\otimes
R_{V,W}).\label{eq:braid_equation} 
\end{align}
\end{proposition}

\begin{proof}
A direct computation shows that
\begin{align*}
(R_{V,W}\otimes\id_{U})(\id_{V}\otimes R_{U,W}) & (R_{U,V}\mathrm{\otimes id}_{W})(u\otimes v\otimes w)\\
 & =\sum(b_{k}b_{j})\cdot w\otimes(a_{k}b_{i})\cdot v\otimes(a_{j}a_{i})\cdot u.
\end{align*}
On the other hand,
\begin{align*}
(\id_{W}\otimes R_{U,V})(R_{U,W}\otimes\id_{V}) & (\id_{U}\otimes R_{V,W})(u\otimes v\otimes w)\\
 & =\sum(b_{j}b_{i})\cdot w\otimes(b_{k}a_{i})\cdot v\otimes(a_{k}a_{j})\cdot u
\end{align*}
and hence the claim follows from propositionosition \ref{proposition:R-matrix}.\end{proof}

\begin{exercise}
\label{exercise:QT_hexagons}
Let $(H,R)$ be a quasitriangular Hopf algebra, and let $U$, $V$ and $W$
be three left $H$-modules. Prove that 
\begin{align}
R_{U\otimes V,W} & =(R_{U,W}\otimes\id_{V})(\id_{U}\otimes R_{V,W}),\label{eq:(Rx1)(1xR)}\\
R_{U,V\otimes W} & =(\id_{V}\otimes R_{U,W})(R_{U,V}\otimes\id_{W}).\label{eq:(1xR)(Rx1)}
\end{align}
\end{exercise}

Setting $U=V=W$ in proposition \ref{proposition:braid_equation} we conclude that
$R_{V,V}$ is a solution of the braid equation for any left $H$-module $V$. 

\begin{definition}
\index{Hopf algebra!almost cocommutative}
\index{almost cocommutative Hopf algebra}
A Hopf algebra $H$ is called \textbf{almost cocommutative} if there exists
an invertible elemmaent $R\in H\otimes H$ such that $\tau\Delta(h)R=R\Delta(h)$
for all $h\in H$.
\end{definition}

\begin{proposition}
Let $(H,R)$ be an almost cocommutative Hopf algebra. Then $S^2$ is an inner
automorphism of $H$. More precisely, assume that $R=\sum a_i\otimes b_i$, and
let $u=\sum (Sb_i)a_i$. Then $u$ is invertible in $H$ and
\[
S^2h=uhu^{-1}=(Su)^{-1}h(Su)
\]
for all $h\in H$.
\end{proposition}

\begin{proof}
First we prove that $uh=(S^2h)u$ for all $h\in H$. Since $H$ is almost cocommutative, 
$(R\otimes1)(h_1\otimes h_2\otimes h_3)=(h_2\otimes h_1\otimes h_3)(R\otimes1)$, i.e.,
\[
\sum a_ih_1\otimes b_ih_2\otimes h_3=\sum h_2a_i\otimes h_1b_1\otimes h_3.
\]
Then
\[
\sum S^2h_3S(b_ih_2)a_ih_1=\sum (S^2h_3)S(h_1b_i)h_2a_i.
\]
Using propositionerties of the antipode and the counit we obtain: 
\[
\sum S^2h_3S(b_ih_2)a_ih_1=\sum S(h_2Sh_3)(Sb_i)a_ih_1= \sum (Sb_i)a_ih=uh.
\]
Similarly, 
\[
\sum (S^2h_3)S(h_1b_i)h_2a_i=\sum S^2h_3(Sb_i)(Sh_1)h_2a_i=\sum (S^2h)(Sb_i)a_i=(S^2h)u
\]
and hence $uh=(S^2h)u$ for all $h\in H$.

Now we prove that $u$ is invertible. Assume that $R^{-1}=\sum c_j\otimes d_j$ and
let $v=\sum S^{-1}(d_j)c_j$. Since $uh=(S^2h)u$, we obtain: 
\begin{align*}
uv=\sum_j u(S^{-1}d_j)c_j&=\sum_j (Sd_j)uc_j\\
&=\sum_{i,j} (Sd_j)(Sb_i)a_ic_j=\sum_{i,j} S(b_id_j)a_ic_j.
\end{align*}
Therefore $uv=1$ since $1\otimes1=RR^{-1}=\sum_{i,j}a_ic_j\otimes b_id_j$.
Using $S^2h=uhu^{-1}$ with $h=v$ we obtain $1=S^2(v)u$ and hence $u$ is
invertible. 

The formula $S^2h=(Su)^{-1}h(Su)$ follows from applying $S$ to
$S^2h=uhu^{-1}$ and replacing $Sh$ by $h$.
\end{proof}

\begin{corollary}
Let $(H,R)$ be an almost cocommutative Hopf algebra. Then the elemmaent $u(Su)$
is central in $H$.
\end{corollary}

\begin{exercise}
\label{exercise:VW=WV}
Let $H$ be an almost cocommutative Hopf algebra. Let $V$ and $W$ be two left
$H$-modules.  Then $V\otimes W\simeq W\otimes V$ as left $H$-modules.
\end{exercise}

%\subsection{Coquasitriangular Hopf algebras}
%
%\begin{definition}
%	A \textbf{coquasitriangular} Hopf algebra is a Hopf algebra $H$ together with a
%	convolution-invertible bilinear form
%	$\langle\cdot|\cdot\rangle:H\otimes H\to\mathbb{K}$ such that 
%	\begin{align*}
%		\langle h_1|k_1\rangle k_2h_2 &= h_1k_1\langle h_2|k_2\rangle,\\
%		\langle h|kl\rangle &= \langle h_1|k\rangle\langle h_2|l\rangle,\\
%		\langle hk|l\rangle &= \langle h|l_2\rangle\langle k|l_1\rangle,
%	\end{align*}
%	for all $h,k,l\in H$.
%\end{definition}
%
%\begin{exercise}
%	Let $H$ be a finite-dimensional Hopf algebra. Prove that $H$ is
%	coquasitriangular if and only if $H^*$ is quasitriangular.
%\end{exercise}
%
%\begin{solution}
%	blah
%\end{solution}
%
%\begin{example}
%	Let $H$ be a commutative Hopf algebra. Then $H$ is coquasitriangular with
%	$\langle h|k\rangle=\varepsilon(h)\varepsilon(k)$.
%\end{example}
%
%\begin{example}
%	Let $G$ be a group and $G=\mathbb{K}[G]$ be the group algebra. Then $H$ is coquasitriangular
%	if and only if $G$ is abelian, 
%	\[
%	\langle h|gl\rangle =\langle h|g\rangle\langle h|l\rangle\quad\text{ and }\quad 
%	\langle hg|l\rangle =\langle h|l\rangle \langle g|l\rangle
%	\]
%	for all $g,h,l\in G$.
%\end{example}
%
%\begin{exercise}
%	Let $H$ be a \textbf{almost commutative} Hopf algebra, i.e.,  
%	\[
%		\langle h_1|k_1\rangle k_2h_2 = h_1k_1\langle h_2|k_2\rangle
%	\]
%	for all $h,k\in H$. Let $V$ and $W$ be two right $H$-comodules. Prove that
%	\[
%	V\otimes W\simeq W\otimes V
%	\]
%	as right $H$-comodules.
%\end{exercise}
%

%\section{Actions and coactions}

\section{(Co)actions on (co)algebras}

\begin{definition}
\label{def:module_algebra}
\index{module-algebra}
Let $H$ be a Hopf algebra. A left \textbf{$H$-module-algebra} is an algebra
$A$ with a left $H$-module structure such that
\begin{align*}
&h\rightarrow(ab)=(h_{1}\rightarrow a)(h_{2}\rightarrow b),\\
&h\rightarrow1=\varepsilon(h)1
\end{align*}
for all $h\in H$ and $a,b\in A$. 
\end{definition}

It is possible to define \textbf{right} $H$-module-algebras: it is an
algebra with a right $H$-module structure such that $(ab)\leftarrow
h=(a\leftarrow h_{1})(b\leftarrow h_{2})$ and $1\cdot h=\varepsilon(h)1$ for
all $h\in H$ and $a,b\in A$. 

\begin{exercise}
Let $H$ be a Hopf algebra.  Prove that $H^*$ is an left $H$-module-algebra
via  $\langle h\rightharpoonup f|x\rangle=\langle f|xh\rangle$ for all $f\in
H^*$, $h,x\in H$.  Similarly, prove that $H^*$ is a right $H$-module-algebra
via $\langle f\leftharpoonup h|x\rangle=\langle f|xh\rangle$.
\end{exercise}

\begin{exercise}
\label{exercise:adjoint}
\index{adjoint representation}
Let $H$ be a Hopf algebra. Define 
\begin{equation}
a\rightarrow x=a_{1}xS(a_{2})\label{eq:left_adjoint_action}
\end{equation}
for all $a,x\in H$. Prove that $(H,\rightarrow)$ is a left $H$-module-algebra.
The representation \ref{eq:left_adjoint_action} is called the \textbf{left
adjoint representation} of $H$. Similarly, prove that the \textbf{right
adjoint action}
\begin{equation}
x\leftarrow a=S(a_{1})xa_{2}\label{eq:right_adjoint_action}
\end{equation}
gives a right module-algebra over $H$.
\end{exercise}

Let $G$ be a group and $\mathbb{K}[G]$ be the corresponding Hopf algebra. Then
the right adjoint action is given by $a\rightarrow x=axa^{-1}$.

\begin{example}
Let $L$ be a Lie algebra and $U(L)$ be the enveloping algebra with
the canonical Hopf algebra structure. The right adjoint action
is given by \[
a\rightarrow x=ax-xa.
\]
\end{example}

\begin{exercise}
\label{exercise:left_smash}
\index{smash product!left}
Let $H$ be a bialgebra and let $(A,\rightarrow)$ be an left $H$-module-algebra.
There exists an algebra structure on $A\otimes H$ given by
\[
(a\otimes h)(b\otimes g)=a(h_{1}\rightarrow b)\otimes h_{2}g
\]
and unit $1\otimes1$. This algebra is called the \textbf{left smash product}
of $A$ and $H$.  Observe that the maps $A\to A\otimes H$, $a\mapsto a\otimes1$,
and $H\to A\otimes H$, $h\mapsto 1\otimes h$ are algebra embedings.
\end{exercise}

\begin{exercise}
\index{smash product!right}
Let $H$ be a Hopf algebra and $(A,\leftarrow)$ be an right $H$-module-algebra.
Prove that there exists an algebra structure on $H\otimes A$ given by 
\[
(h\otimes a)(g\otimes b)=hg_{1}\otimes(a\leftarrow g_{2})b
\]
and unit $1\otimes1$. This algebra is called the \textbf{right
smash product} of $H$ and $A$. 
\end{exercise}

%\section{Actions on coalgebras}

\begin{definition}
\label{def:module_coalgebra}
\index{module-coalgebra}
Let $H$ be a Hopf algebra. A left \textbf{$H$-module-coalgebra}
is a coalgebra $C$ with a left $H$-module structure such that 
\begin{align*}
(h\rightarrow c)_{1}\otimes(h\rightarrow c)_{2} & =(h_{1}\rightarrow c_{1})(h_{2}\rightarrow c_{2}),\\
\varepsilon(h\rightarrow c) & =\varepsilon(h)\varepsilon(c)
\end{align*}
for all $h\in H$ and $c\in C$. 
\end{definition}

A \textbf{right} $H$-module-coalgebra is a coalgebra $C$ with a right
$H$-module structure such that 
\begin{align*}
(c\leftarrow h)_{1}\otimes(c\leftarrow h)_{2} & =(c_{1}\leftarrow h_{1})(c_{2}\leftarrow h_{2})\\
\varepsilon(c\leftarrow h) & =\varepsilon(h)\varepsilon(c)
\end{align*}
for all $h\in H$, $c\in C$.

\begin{exercise}
\index{coadjoint action}
Let $H$ be a finite-dimensional Hopf algebra. Consider the actions
$(a\rightharpoonup f)(b)=f(ba)$ and $(f\leftharpoonup a)(b)=f(ab)$
for all $a,b\in H$, $f\in H^{*}$. The \textbf{left coadjoint action}
of $H$ on $H^{*}$ is 
\[
h\triangleright f=h_{1}\rightharpoonup f\leftharpoonup S^{-1}h_{2}=f(S^{-1}h_{2}?h_1),
\]
where $f(?)$ means the function $x\mapsto f(x)$. Prove that
$(H^*)^\mathrm{cop}$ is a left $H$-module-coalgebra via the left coadjoint
action. Similarly, the \textbf{right coadjoint action} of $H$ on $H^{*}$ is 
\[
f\triangleleft h=S^{-1}h_{1}\rightharpoonup f\leftharpoonup h_{2}=f(h_2?S^{-1}h_{1}).
\]
Prove that $H$ is a right $(H^*)^\mathrm{cop}$-module-coalgebra 
\end{exercise}

\begin{example}
Let $G$ be a finite group and $H=\mathbb{K}G$ be the group Hopf algebra. Then
$y\rightharpoonup e_{x}=e_{xy^{-1}}$ (resp. $e_{x}\leftharpoonup
y=e_{y^{-1}x}$) defines a left (resp. right) $H$-module structure over $H^{*}$.
The left coadjoint action of $H$ over $H^{*}$ is 
\[
y\triangleright e_{x}=y\rightharpoonup e_{x}\leftharpoonup y^{-1}=e_{xyx^{-1}}.
\]
\end{example}

\begin{exercise}
\index{regular action}
Let $H$ be a Hopf algebra and consider the \textbf{left regular action} of $H$ on itself:
$h\rightarrow g=gh$ for all
$h,g\in H$. Prove that $H$ is a left $H$-module-coalgebra. 
\end{exercise}

%\section{Coactions on algebras}

Recall that a \textbf{left $H$-comodule} is a pair $(V,\delta)$,
where $V$ is a vector space and $\delta:V\to H\otimes V$ is a linear
map such that 
\begin{align*}
(\id\otimes\delta)\delta & =(\Delta\otimes\id)\delta,\\
(\varepsilon\otimes\id)\delta & =\id.
\end{align*}
We write $\delta(v)=v_{-1}\otimes v_{0}$. Similarly, a \textbf{right
$H$-comodule} is a pair $(V,\delta)$, where $\delta:V\to V\otimes H$
is a linear map such that 
\begin{align*}
(\id\otimes\Delta)\delta & =(\delta\otimes\id)\delta,\\
(\id\otimes\varepsilon)\delta & =\id.
\end{align*}
In this case we write $\delta(v)=v_{0}\otimes v_{1}$.

\begin{definition}
\index{comodule-algebra}
Let $H$ be a Hopf algebra. An algebra $A$ is a said to be a left
\textbf{$H$-comodule-algebra} if $(A,\delta)$ is a left $H$-comodule and the
following properties are satisfied:
\begin{align*}
\delta(1_{A}) & =1_{H}\otimes1_{A},\\
\delta(ab) & =a_{-1}b_{-1}\otimes a_{0}b_{0}
\end{align*}
for all $a,b\in A$. (Here we write $\delta(a)=a_{-1}\otimes a_{0}\in H\otimes A$.)
\end{definition}

%\section{Coactions on coalgebras}

\begin{definition}
\index{comodule-coalgebra}
Let $H$ be a Hopf algebra. A coalgebra $C$ is said to be a left
\textbf{$H$-comodule-coalgebra} if $(C,\delta)$ is a left $H$-comodule and
the following properties are satisfied:
\begin{align*}
c_{-1}\varepsilon(c_{0}) & =\varepsilon(c)1,\\
(c_{1})_{-1}(c_{2})_{-1}\otimes(c_{1})_{0}\otimes(c_{2})_{0} & =c_{-1}\otimes(c_{0})_{1}\otimes(c_{0})_{2}
\end{align*}
for all $c\in C$.
\end{definition}

\begin{exercise}
\index{coadjoint coaction}
Let $H$ be a Hopf algebra. Consider the \textbf{left coadjoint coaction}
of $H$ on $H$: $\mathrm{coadj}(h)=h_{1}S(h_{3})\otimes h_{2}$ for $h\in H$. Prove that 
$H$ is a left $H$-comodule-coalgebra via the left coadjoint coaction.
\end{exercise}

\begin{exercise}
Let $H$ be a Hopf algebra, $C$ be a coalgebra and $f\in\hom(C,H)$
be a coalgebra map with convolution inverse $g$. Prove that $(C,\delta)$
is a left $H$-comodule coalgebra, where $\delta(c)=f(c_{1})g(c_{3})\otimes c_{2}$
for all $c\in C$. 
\end{exercise}

\begin{exercise}
\label{exercise:smash_coleft}
\index{smash coproduct!left}
Let $H$ be a Hopf algebra, and $(C,\delta)$ be a left $H$-comodule
coalgebra. Prove that $C\otimes H$ is a coalgebra with coproduct
\[
\Delta(c\otimes h)=\left(c_{1}\otimes c_{2,-1}h_{1}\right)\otimes\left(c_{2,0}\otimes h_{2}\right),
\]
and counit $\varepsilon(c\times h)=\varepsilon_{C}(c)\varepsilon_{H}(h)$ for
all $c\in C$, $h\in H$. This coalgebra structure on $C\otimes H$ is called the
\textbf{left smash coproduct}. Observe that the maps $C\otimes H\to C$,
$c\otimes h\mapsto c\varepsilon(h)$, and $C\otimes H\to H$, $c\otimes h\mapsto
\varepsilon(c)h$, are coalgebra surjections.
\end{exercise}

\index{smash coproduct!right}
Assume that $C$ is a right $H$-comodule coalgebra. The \textbf{right}
smash coproduct is then defined by 
\[
\Delta(h\otimes c)=h_{1}\otimes c_{1,0}\otimes h_{2}c_{1,1}\otimes c_{2}
\]
for all $h\in H$ and $c\in C$.


\section{The Drinfeld double}

Now we will construct the Drinfeld double of a finite-dimensional Hopf algebra.
We first need two very well known actions.

\begin{exercise}
Let $C$ be a coalgebra. There exists a natural left action of $C^*$ on $C$
given by $f\rightharpoonup c=\langle f|c_2\rangle c_1$ for all $f\in C^*$ and
$c\in C$. Prove that this action is the transpose of the right
multiplication of $C^*$ on itself, i.e., 
\[
\langle g|f\rightharpoonup c\rangle=\langle f|c_2\rangle\langle g|c_1\rangle=\langle gf|c\rangle
\]
for all $f,g\in C^*$ and $c\in C$. 
Similarly, there is also a natural right action of $C^*$ on $C$ given by
$c \leftharpoonup f=\langle f|c_1\rangle c_2$.  As before, this action is the
transpose of the left multiplication of $C^*$ on itself: 
\[
\langle g|c\leftharpoonup f\rangle=\langle fg|c\rangle
\] 
for all $f,g\in C^*$ and $c\in C$.
\end{exercise}

\begin{exercise}
Let $A$ be an algebra. Then we define a left action of $A$ on $A^*$ which is
the transpose of the right multiplication on $A$: $\langle a\rightharpoonup
f|x\rangle=\langle f|xa\rangle$ for all $f\in A^*$ and $a,x\in A$. 
%Since $\dim A<\infty$, we obtain $a\rightharpoonup f=\langle f_2|a\rangle f_1$. 
Similarly, one can define a right action of $A$ on $A^*$ by $\langle
f\leftharpoonup a|x\rangle = \langle f|ax\rangle$.
%and, as before, $f\leftharpoonup a=\langle f|a\rangle f$.
\end{exercise}

Let $H$ be a Hopf algebra with bijective antipode. The \underline{left
coadjoint action} of $H$ on $H^*$ is the action 
\[
h\triangleright f=h_1\rightharpoonup f\leftharpoonup S^{-1}h_2=f(S^{-1}h_2?h_1)
\]
for all $h\in H$, $f\in H^*$. Notice that $\langle h\triangleright
f|x\rangle=\langle f|S^{-1}h_2xh_1\rangle$. Similarly, one can define the
\underline{right coadjoint action} of $H$ on $H^*$ as 
\[
f\triangleleft h=S^{-1}h_1\rightharpoonup f\leftharpoonup h_2=f(h_2?S^{-1}h_1)
\]
for al $f\in H^*$, $h\in H$. As before, $\langle f\triangleleft h|x\rangle=\langle f|h_2xS^{-1}h_1\rangle$.

\begin{exercise}
Prove that the left coadjoint action of $H$ on $H^*$ is the
transpose of the left adjoint action of $H$ on itself. More
precisely, prove that
\[
\langle h\triangleright f|x\rangle=\langle f|(\mathrm{ad}_l S^{-1}h)(x)\rangle\\
\]
for all $f\in H^*$ and $h,x \in H$,
where $\mathrm{ad}_l(h)(x)=h_1x(Sh_2)$. 
Similarly, prove that 
\[
\langle f\triangleleft h|x\rangle=\langle f|(\mathrm{ad}_r S^{-1}h)(x)\rangle
\]
where $\mathrm{ad}_r(h)(x)=(Sh_1)xh_2$
\end{exercise}
%For that purpose, recall that $(H^*)^\text{cop}$ and $H$ are in duality by
%the evaluation, i.e., there exists a bilinear map
%$(\cdot|\cdot):(H^*)^\text{cop}\to H$ such that: \begin{align*}
%	(f|xy)&=(f_2|x)(f_1|y),\\ (fg|x)&=(f|x_1)(g|x_2),\\ (1|x)&=\varepsilon(x),\\
%	(f|1)&=\varepsilon(f), \end{align*} for all $f,g\in H^*$ and $x,y\in H$.

\begin{exercise}
Assume that $H$ is finite-dimensional. We consider the left coadjoint action of
$H$ on $H^*$ and the right coadjoint action of $H^*$ on $H$. Prove that 
\begin{equation*}
	\Delta^{\mathrm{cop}}(h\triangleright f)=(h_1\triangleright f_2)\otimes(h_2\triangleright f_1)\text{ and }
	\Delta(h\triangleleft f)=(h_1\triangleleft f_2)\otimes(h_2\triangleleft f_1)
\end{equation*}
for all $h\in H$, $f\in H^*$.
\end{exercise}

\begin{theorem}
\label{theorem:double}
Let $H$ be a finite dimensional Hopf algebra. The \underline{Drinfeld double}
$\mathcal{D}(H)$ of $H$ is a Hopf algebra. It can be realized on the vector
space $(H^{*})^{\text{cop}}\otimes H$ with product
\begin{align*}
(f\otimes h)(f'\otimes h')&=ff'_{2}\otimes h_{2}h'\langle f'_{3}|h_{1}\rangle\langle f'_{1}|S^{-1}h_{3}\rangle\\
&=f(h_1\rightharpoonup f'\leftharpoonup S^{-1}h_3)\otimes h_2h'\\
&=f(h_1\triangleright f_2')\otimes (h_2\triangleleft f_1')h',
\end{align*}
unit $1\otimes1$, coproduct 
\[
\Delta(f\otimes h)=f_{2}\otimes h_{1}\otimes f_{1}\otimes h_{2},
\]
counit $\varepsilon(f\otimes h)=\varepsilon(f)\varepsilon(h)$ and
antipode 
\begin{align*}
S(f\otimes h)&=(Sh_2\rightharpoonup Sf_1)\otimes (f_2\rightharpoonup Sh_1)\\
&=(Sf_2\leftharpoonup h_1)\otimes(Sh_2\leftharpoonup Sf_1)
\end{align*}
for $f,f'\in H^*$ and $h,h'\in H$. 
\end{theorem}

\begin{exercise}
Prove Theorem \ref{theorem:double}.
\end{exercise}

%\begin{proof}
%We prove that $\Delta$ is morphism of algebras. 
%We compute:
%\begin{align*}
%	\Delta\left( (f\otimes h)(f'\otimes h') \right) &= (f'_3|h_1)(f'_1|S^{-1}h_3)\Delta( ff'_2\otimes h_2h')\\
%	&= (f'_3|h_1)(f'_1|S^{-1}h_3)(ff'_2)_2\otimes (h_2h')_1\otimes (ff'_2)_1\otimes (h_2h')_2\\
%	&= (f'_4|h_1)(f'_1|S^{-1}h_4)f_2f'_3\otimes h_2h'_1\otimes f_1f'_2\otimes h_3h'_2.
%\end{align*}
%On the other hand,
%\begin{align*}
%	\Delta(f&\otimes h)\Delta(f'\otimes h') = (f_2\otimes h_1)(f'_2\otimes h'_1)\otimes (f_1\otimes h_2)(f'_1\otimes h'_2)\\
%	&= (f'_5|h_1)(f'_{32}|S^{-1}h_{31})(f'_{31}|h_{32})(f'_1|S^{-1}h_5)f_2f'_4\otimes h_2h'_1\otimes f_1f'_2\otimes h_4h'_2\\
%	&= (f'_5|h_1)(f'_{3}|S^{-1}(h_{31})h_{32})(f'_1|S^{-1}h_5)f_2f'_4\otimes h_2h'_1\otimes f_1f'_2\otimes h_4h'_2\\
%	&= (f'_4|h_1)(f'_1|S^{-1}h_4)f_2f'_3\otimes h_2h'_1\otimes f_1f'_2\otimes h_3h'_2.
%\end{align*}
%The rest is left as an exercise.
%\end{proof}

\begin{exercise}
Prove that  the product of $\mathcal{D}(H)$ is: 
\[
(f\otimes h)(f'\otimes h') = ff'(S^{-1}(h_{3})?h_{1})\otimes h_{2}h'
\]
where $f(?)$ means the map $x\mapsto f(x)$. 
\end{exercise}

%\begin{exercise}
%Prove that  the product of $\mathcal{D}(H)$ is: 
%\begin{align*}
%(f\otimes h)(f'\otimes h') &= f(h_{1}\rightarrow f'\leftarrow S^{-1}h_{3})\otimes h_{2}h'\\
%&=ff'_{2}\otimes(S^{-1}f'_{1}\to h\leftarrow f'_{3})h',
%\end{align*}	
%where 
%$f\leftarrow h=f(h?)$,     
%$h\rightarrow f=f(?h)$,        
%$f\to h=f(h_{2})h_{1}$ and 
%$h\leftarrow f=f(h_1)h_2$
%for all $f\in H^*$ and $h\in H$. 
%\end{exercise}

\begin{exercise}
Let $H$ be a finite-dimensional cocommutative Hopf algebra. Prove that
$\mathcal{D}(H)$ is isomorphic (as an algebra) to the smash product on
$H^{*}\otimes H$, see \cite[10.3.10]{MR1243637}.
\end{exercise}

\begin{lemma}
Let $H$ be a finite-dimensional.
Assume that $\{h_{i}\}$ is a basis of $H$ and $\{h^{i}\}$ is a basis
of $H^{*}$ dual to $\{h_{i}\}$. Then 
\begin{equation}
	R=\sum_{i}(\varepsilon\otimes h_{i})\otimes(h^{i}\otimes1)
\end{equation}
does not depend on $\{h_i\}$ and $\{h^i\}$.
\end{lemma}

\begin{proof}
Since $H$ is finite-dimensional, the linear map $\Phi:H\otimes
H^{*}\to\mathrm{End}_{\mathbb{K}}(H)$ defined by $\Phi(h\otimes f)(x)=f(x)h$ is
an isomorphism. We prove that $\Phi^{-1}(\id)=\sum h_{i}\otimes h^{i}$
does not depend on the pair of dual basis $\{h_{i}\}$ and $\{h^{i}\}$:
\[
\Phi(\sum h_{i}\otimes h^{i})(x)=\sum\Phi(h_{i}\otimes h^{i})(x)=\sum h^{i}(x)h_{i}=x.
\]
Since $R=\varepsilon\otimes\Phi^{-1}(\id)\otimes1$, the claim
follows. 
\end{proof}

\begin{theorem}
\label{theorem:R_matrix}
Let $H$ be a finite-dimensional Hopf algebra. Then $\mathcal{D}(H)$ is a
quasitriangular Hopf algebra. More precisely, the quasitriangular structure is
given by 
\begin{equation}
R=\sum_{i}(\varepsilon\otimes h_{i})\otimes(h^{i}\otimes1),\label{eq:R}
\end{equation}
where $\{h_{i}\}$ is a basis of $H$ and $\{h^{i}\}$ is a basis
of $H^{*}$ dual to $\{h_{i}\}$. 
\end{theorem}

\begin{exercise}
Prove Theorem \ref{theorem:R_matrix}.
\end{exercise}

\begin{corollary}
Let $H$ be a finite-dimensional Hopf algebra. Then $H$ is a subHopf algebra of
a quasitriangular Hopf algebra. 
\end{corollary}

\begin{proof}
It follows from the fact that $H\simeq\varepsilon_{H}\otimes H$ is a subalgebra
of $\mathcal{D}(H)$. 
\end{proof}

\begin{example}
Let $G$ be a finite group, and let $H=\mathbb{K}[G]$ be the group algebra of
$G$ with the usual Hopf algebra structure. Let $\{e_{g}\mid g\in G\}$ be the
dual basis of the basis $\{g\mid g\in G\}$ of $H$. The dual algebra
$\left(\mathbb{K}[G]^{\text{op}}\right)^{*}$ is the algebra
$\mathrm{Fun}(G,\mathbb{K})$ with multiplication
\[
e_{g}e_{h}=\begin{cases}
e_{g} & \text{if }g=h,\\
0 & \text{otherwise,}
\end{cases}
\]
for all $g,h\in G$ and unit $\sum_{g\in G}e_{g}=1$. 
The comultiplication
is 
\[
\Delta(e_{g})=\sum_{uv=g}e_{v}\otimes e_{u},
\]
the counit is 
\[
\varepsilon(e_{g})=\begin{cases}
1 & \text{if }g=1,\\
0 & \text{otherwise,}
\end{cases}
\]
and the antipode is 
$S(e_{g})=e_{g^{-1}}$
for all $g\in G$.  Now we describe
the Drinfeld double $\mathcal{D}(\mathbb{K}[G])$.  A basis of
$\mathcal{D}(\mathbb{K}[G])$ is given by
\[
\{e_gh\mid (g,h)\in G\times G\}.
\]
The product of $\mathcal{D}(\mathbb{K}[G])$ is determined by 
\[
he_{g}=e_{h^{-1}gh}h.
\]
The $R$-matrix is
\[
R=\sum_{g\in G}g\otimes e_{g}.
\]
\end{example}


\section{Yetter-Drinfeld modules}

\begin{definition}
\index{Yetter-Drinfeld module}
Let $H$ be a Hopf algebra. A \underline{Yetter-Drifeld module} over $H$ is a
triple $(V,\rightarrow,\delta)$, where $(V,\rightarrow)$ is a left $H$-module,
$(V,\delta)$ is a left $H$-comodule, and such that 
\begin{equation}
\delta(h\rightarrow v)=h_{1}v_{-1}Sh_{3}\otimes h_{2}\rightarrow v_{0}\label{eq:YD}
\end{equation}
for all $h\in H$, $v\in V$. A \underline{morphism} of Yetter-Drinfeld modules
is a morphism of left $H$-modules and left $H$-comodules. The category of Yetter-Drinfeld
modules will be denoted by $\ydH$.
\end{definition}

\begin{example}
Let $H$ be a Hopf algebra with the trivial action and coaction on itself:
$h\rightarrow x=\varepsilon(h)x$ and $\delta(h)=1\otimes h$ for all $h,x\in H$.
Then $(H,\rightarrow,\delta)$ is a Yetter-Drinfeld module over $H$.
\end{example}

\begin{example}
Let $H$ be a Hopf algebra. Then $(H,\mathrm{adj},\Delta)$ and
$(H,m,\mathrm{coadj})$ are Yetter-Drinfeld modules over $H$.
\end{example}

\begin{exercise}
\label{exercise:YD_condition}
Prove that the condition \eqref{eq:YD} is equivalent to
\begin{equation}
\label{eq:left_left_YD_equivalent}
h_{1}v_{-1}\otimes(h_{2}\rightarrow v_{0})=(h_{1}\rightarrow v)_{-1}h_{2}\otimes(h_{1}\rightarrow v)_{0}
\end{equation}
for all $h\in H$, $v\in V$.
\end{exercise}

\begin{exercise}
Let $G$ be a group, and $H$ be the group Hopf algebra of $G$. Assume that
$(V,\rightarrow)$ is a left $H$-module, and $(V,\delta)$ is a left
$H$-comodule. 
\begin{enumerate}
\item Prove that $V=\oplus_{g\in G}V_{g}$, where $V_{g}=\{v\in V\mid\delta(v)=g\otimes v\}$.  
\item Prove that the triple $(V,\rightarrow,\delta)$ is a Yetter-Drinfeld
module if and only if $h\rightarrow V_{g}\subseteq V_{hgh^{-1}}$ for all
$g,h\in H$.
\end{enumerate}
\end{exercise}

\begin{exercise}\label{exercise:YD_tensor}
Let $V$ and $W$ be two Yetter-Drinfeld modules over $H$. Then $V\otimes W$ is a
Yetter-Drinfeld over $H$, where 
\begin{align*}
h\rightarrow(v\otimes w) & =(h_{1}\rightarrow v)\otimes(h_{2}\rightarrow w),\\
\delta(v\otimes w) & =v_{-1}w_{-1}\otimes(v_{0}\otimes w_{0})
\end{align*}
for all $h\in H$, $v\in V$, $w\in W$.
\end{exercise}

Let $H$ be a Hopf algebra with invertible antipode. For any pair $V$ and $W$ of
Yetter-Drinfeld modules over $H$, we consider the map 
\begin{align*}
c_{V,W}:V\otimes W&\to W\otimes V\\
v\otimes w&\mapsto (v_{-1}\rightarrow w)\otimes v_{0}.
\end{align*}

\begin{lemma}
The map $c_{V,W}$ is an isomorphism in $\ydH$.
\end{lemma}

\begin{proof}
The map $c$ is invertible and the inverse is 
\begin{align*}
c_{V,W}^{-1}:W\otimes V & \to V\otimes W\\
w\otimes v & \mapsto v_{0}\otimes(S^{-1}(v_{-1})\to w)
\end{align*}
since
\begin{align*}
c_{V,W}^{-1}c_{V,W}(v\otimes w) & =c_{V,W}^{-1}((v_{-1}\to w)\otimes v_{0})\\
 & =v_{0,0}\otimes(S^{-1}(v_{0,-1})\to(v_{-1}\to w))\\
 & =v_{0,0}\otimes(S^{-1}(v_{0,-1})v_{-1}\to w)\\
 & =v_{0}\otimes(S^{-1}(v_{-1})v_{-2}\to w)\\
 & =v_{0}\otimes(\varepsilon(v_{-1})1\to w)\\
 & =v\otimes w,
\end{align*}
and similarly $c_{V,W}c_{V,W}^{-1}(w\otimes v)=w\otimes v$. 

Now we prove that $c_{V,W}$ is a morphism of $H$-modules: 
\begin{align*}
c_{V,W}(h\rightarrow (v\otimes w))&=c_{V,W}(h_1\rightarrow v\otimes h_2\rightarrow w)\\
&=(h_1\rightarrow v)_{-1}\rightarrow(h_2\rightarrow w)\otimes(h_1\rightarrow v)_0\\
&=(h_{11}v_{-1}Sh_{13})\rightarrow(h_2\rightarrow w)\otimes h_{12}\rightarrow v_0\\
&=(h_1v_{-1}(Sh_3)h_4)\rightarrow w\otimes h_2\rightarrow v_0\\
&=(h_1v_{-1})\rightarrow w\otimes h_2\rightarrow v_0\\
&=h_1\rightarrow(v_{-1}\rightarrow w)\otimes h_2\rightarrow v_0\\
&=h\rightarrow((v_{-1}\rightarrow w)\otimes v_0).
\end{align*}

To prove that $c_{V,W}$ is a morphism of comodules we need $(\id\otimes
c)\delta=\delta c$.  We compute:
\[
(\id\otimes c)\delta(v\otimes w)=v_{-1}w_{-1}\otimes (v_{0,-1}\rightarrow w_0)\otimes v_{0,0}.
\]
On the other hand,
\begin{align*}
\delta(c(v\otimes w)&=\delta(v_{-1}\rightarrow w\otimes v_0)\\
&=(v_{-1}\rightarrow w)_{-1}v_{0,-1}\otimes(v_{-1}\rightarrow w)_0\otimes v_{0,0}\\
&=(v_{-2}\rightarrow w)_{-1}v_{-1}\otimes (v_{-2}\rightarrow w)_0\otimes v_0\\
&=v_{-2,1}w_{-1}S(v_{-2,3})v_{-1}\otimes(v_{-2,2}\rightarrow w_0)\otimes v_0\\
&=v_{-4}w_{-1}S(v_{-2})v_{-1}\otimes (v_{-3}\rightarrow w_0)\otimes v_0\\
&=v_{-2}w_{-1}\otimes(v_{-1}\rightarrow w_0)\otimes v_0.
\end{align*}
This completes the proof.
\end{proof}

\begin{exercise}
\label{exercise:YD_hexagons}
Let $H$ be a Hopf algebra, and let $U$, $V$ and $W$
be three objects of $_H^H\mathcal{YD}$. Prove that 
\begin{align}
c_{U\otimes V,W} & =(c_{U,W}\otimes\id_{V})(\id_{U}\otimes c_{V,W}),\label{eq:(cx1)(1xc)}\\
c_{U,V\otimes W} & =(\id_{V}\otimes c_{U,W})(c_{U,V}\otimes\id_{W}).\label{eq:(1xc)(cx1)}
\end{align}
\end{exercise}

\begin{exercise}
\label{exercise:YD_naturality}
Let $H$ be a Hopf algebra. Prove that 
\[
c_{V',W'}(f\otimes g)=(g\otimes f)c_{W,V}
\]
for all Yetter-Drinfeld modules morphisms $f:V\to V'$ and $g:W\to W'$. 
\end{exercise}

\begin{theorem}
\label{theorem:YD_braid_equation}
Let $H$ be a Hopf algebra with invertible antipode, and let $U,V,W$
be Yetter-Drinfeld modules over $H$. Then 
%\begin{align*}
%c_{V,W}:V\otimes W&\to W\otimes V\\
%v\otimes w&\mapsto (v_{-1}\rightarrow w)\otimes v_{0},
%\end{align*}
%is an isomorphism in $_{H}^{H}\mathcal{YD}$ and it is a solution of the braid
%equation:
\begin{align*}
(c_{V,W}\otimes\textrm{id}_{U})(\textrm{id}_{V}&\otimes c_{U,W})(c_{U,V}\otimes\textrm{id}_{W})\\
&=(\textrm{id}_{W}\otimes c_{U,V})(c_{U,W}\otimes\textrm{id}_{V})(\textrm{id}_{U}\otimes c_{V,W}).
\end{align*}
\end{theorem}

\begin{proof}
It follows from Exercise \eqref{exercise:YD_naturality} with
$f=c_{U,V}\otimes\id_W$ and $g=\id_W$, and Exercise
\eqref{exercise:YD_hexagons}.
\end{proof}

\begin{exercise}
Prove Theorem \ref{theorem:YD_braid_equation} without using Exercises
\ref{exercise:YD_naturality} and \ref{exercise:YD_hexagons}.
\end{exercise}

%It remains
%to prove that $c$ is natural, i.e., 
%\[
%(g\otimes f)c_{V,W}=c_{V',W'}(f\otimes g).
%\]
%So let $V$, $V'$, $W$ and $W'$ be objects of $_{H}^{H}\mathcal{YD}$,
%and $f\in\hom(V,V')$, $g\in\hom(W,W')$. We compute 
%\begin{align*}
%(g\otimes f)c_{V,W}(v\otimes w) & =(g\otimes f)(v_{-1}\to w\otimes v_{0})\\
% & =g(v_{-1}\to w)\otimes f(v_{0})\\
% & =v_{-1}\to g(w)\otimes f(v_{0})
%\end{align*}
%and
%\begin{align*}
%c_{V',W'}(f\otimes g)(v\otimes w) & =f(v)\otimes g(w)\\
% & =f(v)_{-1}\to g(w)\otimes f(v)_{0}\\
% & =v_{-1}\to g(w)\otimes f(v_{0})
%\end{align*}
%(here we use that $f$ is morphism of $H$-comodules). Hence the claim
%holds.

\subsection{The category $_H\mathcal{YD}^H$}

We will also work with the following variation of what a Yetter-Drinfeld module
is: An object $V$ in the category $_{H}\mathcal{YD}^{H}$ is a triple
$(V,\rightarrow,\delta)$, where $(V,\rightarrow)$ is a left $H$-module,
$(V,\delta)$ is a right $H$-comodule, such that
\[
h_{1}\rightarrow v_{0}\otimes h_{2}v_{1}=(h_{2}\rightarrow v)_{0}\otimes(h_{2}\rightarrow v)_{1}h_{1},
\]
or equivalently
\[
\delta(h\rightarrow v)=h_{2}\rightarrow v_{0}\otimes h_{3}v_{1}S^{-1}h_{1},
\]
for all $v\in V$, $h\in H$. 
%\end{rem}

%\begin{exercise}
%Let $H$ be a Hopf algebra with bijective antipode. Prove that the categories
%$_{H}^{H}\mathcal{YD}$ and $_{H}\mathcal{YD}^{H}$ are equivalent.
%\end{exercise}
%
%\begin{solution}
%Let $(V,\rightarrow,\delta)$ be an object of $_{H}^{H}\mathcal{YD}$, where we
%write $\delta(v)=v_{-1}\otimes v_{0}$. Let $\rho:V\to V\otimes H$ be the linear
%map defined by $\rho(v)=Sv_{1}\otimes v_{0}$. Then $(V,\rightarrow,\rho)$ is an
%object of $_{H}\mathcal{YD}^{H}$. The converse is similar. 
%\end{solution}

\begin{exercise}
Let $H$ be a finite-dimensional Hopf algebra with bijective antipode.  Assume that
$(V,\rightarrow,\delta_R)$ is an object of $_{H}\mathcal{YD}^{H}$ and define
\[
\delta_L(v)=S(v_1)\otimes v_0
\]
for all $v\in V$.  Prove that
$(V,\rightarrow,\delta_L)$ is an object of $_H^H\mathcal{YD}$. 
Conversely, if $(V,\rightarrow,\delta_L)$ is an object of $_H^H\mathcal{YD}$,
define \[
\delta_R(v)=v_0\otimes S^{-1}v_{-1}
\]
for all $v\in V$. Prove that
$(V,\rightarrow,\delta_R)$ is an object of $_H\mathcal{YD}^H$.
\end{exercise}

\subsection{Yetter-Drinfeld modules and the Drinfeld double}

\begin{exercise}
Let $H$ be a finite-dimensional Hopf algebra. Assume that $\{h_i\}$ is a basis
of $H$, and let $\{h^i\}$ be its dual basis.  Prove that the element
\[
\sum h^i\otimes h_i
\]
does not depend on the pair of dual basis $\{h_i\}$ and $\{h^i\}$.
\end{exercise}

\begin{lemma}
\label{lem:DH_compatibility}
Let $H$ be a finite-dimensional Hopf algebra. Then  $V$ is a left
$\mathcal{D}(H)$-module if and only if $V$ is a left $H$-module, a left
$H^{*}$-module and 
\begin{eqnarray}
h\cdot(f\cdot v) & = & f(S^{-1}(h_{3})?h_{1})\cdot(h_{2}\cdot v)\label{eq:compatibility_D(H)}
\end{eqnarray}
for all $h\in H$, $f\in H^{*}$.
\end{lemma}

\begin{proof}
We compute 
\begin{align*}
(1\otimes h)\cdot((f\otimes1)\cdot v) & =((1\otimes h)(f\otimes1))\cdot v\\
 & =(f(S^{-1}(h_{3})?h_{1})\otimes h_{2})\cdot v\\
 & =(f(S^{-1}(h_{3})?h_{1})\otimes1)(1\otimes h_{2}))\cdot v\\
 & =f(S^{-1}(h_{3})?h_{1})\cdot(h_{2}\cdot v).
\end{align*}
and the claim follows. 
\end{proof}

\begin{lemma}
\label{lem:DH_to_YD}
Let $H$ be a finite-dimensional Hopf algebra and assume that $\{h_i\}$ is a basis
of $H$, and let $\{h^i\}$ be its dual basis.  
Let $(V,\cdot)$ be a left $\mathcal{D}(H)$-module. For any $v\in V$ define 
\[
\delta(v)=\sum h^i\cdot v\otimes h_i.
\]
Then the triple $(V,\cdot,\delta)$ is an object of $_H\mathcal{YD}^H$.
\end{lemma}

\begin{proof}
We prove the compatibility condition
\begin{equation}
\label{eq:DH_to_YD}
\sum h^i\cdot (v\cdot v)\otimes h_i=\sum x_2\cdot(h^i\cdot v)\otimes x_3h_iS^{-1}x_1
\end{equation}
for all $x\in H$, $v\in V$. Let $f\in H^*$ and apply $(\id\otimes f)$ to the
left hand side of \eqref{eq:DH_to_YD} to obtain
\[
\sum h^i\cdot (x\cdot v)f(h_i)=f\cdot (x\cdot v).
\]
On the other hand, applying $(\id\otimes f)$ to the right hand side of
\eqref{eq:DH_to_YD} we obtain
\begin{align*}
%(\id\otimes f)&\left(\sum x_2\cdot (h^i\cdot v)\otimes x_3h_1S^{-1}x_1\right)\\
\sum x_2\cdot(h^i\cdot v) f(x_3h_iS^{-1}x_1)&=
x_2\cdot\left( f(x_3?S^{-1}x_1)\cdot v\right)\\
&=f(x_3S^{-1}x_{23}?x_{21}S^{-1}x_1)\cdot (x_{22}\cdot v)\\
&=f(x_5S^{-1}x_4?x_2S^{-1}x_1)\cdot(x_3\cdot v)\\
&=f\cdot (x\cdot v)
\end{align*}
and the claim follows.
\end{proof}

\begin{lemma}
\label{lem:YD_to_DH}
Let $H$ be a finite-dimensional Hopf algebra. Let 
$(V,\cdot,\delta)$ be an object of $_H\mathcal{YD}^H$. Then $V$ is a left
$\mathcal{D}(H)$-module via 
\[
(f\otimes h)\cdot v=\langle f\mid (h\cdot v)_1\rangle (h\cdot v)_0
\]
for all $f\in H^*$, $h\in H$ and $v\in V$.
\end{lemma}

\begin{proof}
By Lemma \ref{lem:DH_compatibility}, we need prove that
\[
h \cdot(f\cdot v)=\langle f\mid v_1\rangle(h\cdot v_0)
\]
for all $f\in H^*$, $h\in H$, $v\in V$. We compute:
\begin{align*}
f(S^{-1}h_3?h_1)\cdot(h_2\cdot v)&=\langle f\mid S^{-1}h_3(h_2\cdot v)_1h_1\rangle(h_2\cdot v)_0\\
&=\langle f\mid S^{-1}h_3(h_{23}v_1S^{-1}h_{21})h_1\rangle(h_{22}\cdot v_0)\\
&=\langle f\mid S^{-1}h_5h_4v_1S^{-1}h_2h_1\rangle h_3\cdot v_0\\
&=\langle f\mid v_1\rangle (h\cdot v_0)
\end{align*}
and the claim follows.
\end{proof}

\begin{theorem}
The categories $_H\mathcal{YD}^H$ and $_{\mathcal{D}(H)}\mathcal{M}$ are
equivalent.
\end{theorem}

\begin{proof}
It follows from Lemmas \ref{lem:DH_to_YD} and \ref{lem:YD_to_DH}.
\end{proof}


\section{Monoidal categories}

\begin{definition}
\index{monoidal category}
A \textbf{monoidal category} is a tuple
$(\mathcal{C},\otimes,a,\mathbb{I},l,r)$, where $\mathcal{C}$ is a category,
$\otimes:\mathcal{C}\times\mathcal{C}\to\mathcal{C}$ is a funtor, $\mathbb{I}$
is an object of $\mathcal{C}$, $a_{U,V,W}:(U\otimes V)\otimes W\to
U\otimes(V\otimes W)$ is a natural isomorphism such that 
\begin{equation}
(\id_{U}\otimes a_{V,W,X})a_{U,V\otimes W,X}(a_{U,V,W}\otimes\id_{X})=a_{U,V,W\otimes X}a_{U\otimes V,W,X}\label{eq:pentagon}
\end{equation}
for all objects $U$,$V$, $W$ of $\mathcal{C}$ and $r_{U}:U\otimes\mathbb{I}\to U$
and $l_{U}:\mathbb{I}\otimes U\to U$ are natural isomorphism such
that 
\begin{equation}
(\id_{V}\otimes l_{W})a_{V,I,W}=r_{V}\otimes\id_{W}\label{eq:triangles}
\end{equation}
for all objects $U,W$ of $\mathcal{C}$.
\end{definition}

\begin{definition}
\index{strict monoidal category}
A monoidal category $\mathcal{C}$ is called \textbf{strict} if the natural
isomorphism $a$, $l$ y $r$ are identities. 
\end{definition}

\begin{theorem}
Every monoidal category $\mathcal{C}$ is equivalent to a strict monoidal
category.
\end{theorem}

\begin{proof}
See for example \cite[Theorem XI.5.3]{MR1321145}.
\end{proof}

\begin{example}
\index{tensor product!of $H$-modules}
Let $H$ be a Hopf algebra. The category of left $H$-modules is a monoidal
category.  Recall that if $V$ and $W$ are two left $H$-modules, the tensor
product of $V$ and $W$ is defined by 
\[
h\rightarrow(v\otimes w)=(h_{1}\rightarrow v)\otimes(h_{2}\rightarrow w)
\]
for all $h\in H$, $v\in V$, $w\in W$. 
\end{example}

\begin{example}
\index{tensor product!of $H$-comodules}
Let $H$ be a Hopf algebra. The category of left $H$-comodules is a monoidal
category.  Recall that if $V$ and $W$ are two left $H$-comodules, the tensor
product of $V$ and $W$ is defined defined by 
\[
\delta(v\otimes w)=v_{-1}w_{-1}\otimes(v_0\otimes w_0)
\]
for all $v\in V$, $w\in W$.
\end{example}

\begin{example}
\index{tensor product!of Yetter-Drinfeld modules}
Let $H$ be a Hopf algebra with invertible antipode. The category
$_H\mathcal{YD}^H$ of Yetter-Drinfeld modules is a monoidal category. 
\end{example}

%\subsection{Braided categories}

\begin{definition}
\index{braided monoidal category}
A monoidal category $\mathcal{C}$ is \textbf{braided} if there
exists a natural isomorphism $c:\otimes\to\otimes^{\mathrm{op}}$
such that
\begin{align}
c_{U,V\otimes W} & =(\id_{V}\otimes c_{U,W})(c_{U,V}\otimes\id_{W}),\label{eq:braided1}\\
c_{U\otimes V,W} & =(c_{U,W}\otimes\id_{V})(\id_{U}\otimes c_{V,W})\label{eq:braided2}
\end{align}
for all objects $U,V,W$ of $\mathcal{C}$. 
\end{definition}

\begin{definition}
\index{symmetric monoidal category}
A braided monoidal category is \textbf{symmetric} if $c$ satisfies
\[
c_{U,V}c_{V,U}=\id_{U\otimes V}
\]
for all objects $U,V$ of $\mathcal{C}$.
\end{definition}

\begin{remark}
The naturality of the braiding $c$ means that if $V,W$ are objects
of $\mathcal{C}$ then there exists a morphism $c_{V,W}:V\otimes W\to W\otimes V$
such that the diagram 
\[
\xymatrix{V\otimes W \ar[d]_{f\otimes g} \ar[r]^{c_{V,W}} & W\otimes V \ar[d]^{g\otimes f} \\ V'\otimes W' \ar[r]_{c_{V',W'}} & W'\otimes V' }
\]
is commutative for all pair of morphisms $f:V\to V'$ y $g:W\to W'$.
\end{remark}

\begin{proposition}
Let $U$, $V$ and $W$ be objects of a braided monoidal category $\mathcal{C}$.
Then 
\begin{align*}
(c_{V,W}\otimes\textrm{id}_{U})(\textrm{id}_{V}&\otimes c_{U,W})(c_{U,V}\otimes\textrm{id}_{W})\\
&=(\textrm{id}_{W}\otimes c_{U,V})(c_{U,W}\otimes\textrm{id}_{V})(\textrm{id}_{U}\otimes c_{V,W}).
\end{align*}
\end{proposition}

\begin{proof}
It follows from Equations \eqref{eq:braided1}--\eqref{eq:braided2} and the
diagram
\[
\xymatrix{(U\otimes V)\otimes W \ar[r]^{c_{U\otimes V,W}}\ar[d]^{c_{U,V}\otimes\text{id}_W} & W\otimes (U\otimes V)\ar[d]^{\text{id}_W\otimes c_{U,V}} \\ (V\otimes U)\otimes W\ar[r]_{c_{V\otimes U,W}} & W\otimes (V\otimes U)}
\]
obtained from the naturality of the braiding with
$f=c_{U,V}\otimes\id_W$ and $g=\id_W$.
\end{proof}

%\begin{example}
%Let $H$ be a quasitriangular Hopf algebra. The category of left $H$-modules is
%a braided monoidal category.
%\end{example}
\begin{example}
The category $_H^H\mathcal{YD}$ of Yetter-Drinfeld modules is a braided
monoidal category.
\end{example}

\begin{proposition}
Let $H$ be a Hopf algebra. Then $H$ is quasitriangular if and only if
$\lmod{H}$ is a braided monoidal category.
\end{proposition}

\begin{proof}
We first prove the implication $\implies$. Assume that $H$ is quasitriangular
with $R=\sum a_{i}\otimes b_{i}$. Let $V$ and $W$ be two left $H$-modules, and
define
\begin{align*}
c_{V,W}:V\otimes W & \to W\otimes V\\
v\otimes w & \mapsto\sum(b_{i}\cdot w)\otimes(a_{i}\cdot v)
\end{align*}
Since $R$ is invertible, we assume that $R^{-1}=\sum a'_{i}\otimes b'_{i}$. 
Then $c_{V,W}$ is invertible with inverse
\begin{align*}
c_{V,W}^{-1}:W\otimes V & \to V\otimes W\\
w\otimes v & \mapsto\sum(a'_{i}\cdot v)\otimes(b'_{i}\cdot w)
\end{align*}
For example, we check that $c_{V,W}^{-1}\circ c_{V,W}=\id_{V\otimes W}$:
\begin{align*}
(c_{V,W}^{-1}\circ c_{V,W})(v\otimes w) & =\sum c_{V,W}^{-1}((b_{i}\cdot w)\otimes(a_{i}\cdot v))\\
 & =\sum(a'_{j}\cdot a_{i}\cdot v)\otimes(b'_{j}\cdot b_{i}\cdot w)\\
 & =(1\cdot v)\otimes(1\cdot w)\\
 & =v\otimes w.
\end{align*}
Similarly we prove that $c_{V,W}\circ c_{V,W}^{-1}=\id_{W\otimes V}$.
By Lemma \ref{paragraph:QT_braiding}, the map $c_{V,W}$ is a morphism of left $H$-modules. 
%Now we prove that $c_{V,W}$ is a morphism of left $H$-modules:
%\begin{align*}
%c_{V,W}(h\to(v\otimes w)) & =c_{V,W}(h_{1}\to v\otimes h_{2}\to w)\\
% & =\sum b_{i}\to h_{2}\to w\otimes a_{i}\to h_{1}\to v\\
% & =\sum b_{i}h_{2}\to w\otimes a_{i}h_{1}\to v\\
% & =\sum h_{1}b_{i}\to w\otimes h_{2}a_{i}\to v\qquad\mathrm{(by \ref{QT:1})}\\
% & =\sum h_{1}\to(b_{i}\to w)\otimes h_{2}\to(a_{i}\to v)\\
% & =h\to\sum(b_{i}\to w)\otimes(a_{i}\to v)\\
% & =h\to c_{V,W}(v\otimes w).
%\end{align*}
We need to to prove that $c_{V,W}$ is a braiding. First we prove
that $c$ is natural, i.e., 
\[
(g\otimes f)c_{V,W}=c_{V',W'}(f\otimes g)
\]
holds for all $f:V\to V'$ and $g:W\to W'$ any two left $H$-module morphisms.
We compute: 
\begin{align*}
(g\otimes f)c_{V,W}(v\otimes w) & =(g\otimes f)\left(\sum b_{i}\cdot w\otimes a_{i}\cdot v\right)\\
 & =\sum g(b_{i}\cdot w)\otimes f(a_{i}\cdot v)\\
 & =\sum b_{i}\cdot g(w)\otimes a_{i}\cdot f(v)
\end{align*}
and on the other hand,
\begin{align*}
c_{V',W'}(f\otimes g)(v\otimes w) & =c_{V',W'}(f(v)\otimes g(w))\\
 & =\sum b_{i}\cdot g(w)\otimes a_{i}\cdot f(v).
\end{align*}
To prove Equations \eqref{eq:braided1} and \eqref{eq:braided2} we refer to
Exercise \eqref{exercise:QT_hexagons}.
%Now we prove that \eqref{eq:braided1} holds: 
%\begin{align*}
%c_{U,V\otimes W}(u\otimes v\otimes w) & =\sum b_{i}\to(v\otimes w)\otimes a_{i}\to u\\
% & =\sum b_{i,1}\to v\otimes b_{i,2}\to w\otimes a_{i}\to u\\
% & =\sum b_{j}\to v\otimes b_{i}\to w\otimes a_{i}a_{j}\to u\qquad\mathrm{(by \ref{QT:2})}\\
% & =\sum b_{j}\to v\otimes b_{i}\to w\otimes a_{i}\to(a_{j}\to u)
%\end{align*}
%and 
%\begin{align*}
%(\id_{V}\otimes c_{U,W})(c_{U,V}\otimes\id_{W}) & (u\otimes v\otimes w)\\
%= & \sum(\id_{V}\otimes c_{U,W})(b_{j}\to v\otimes a_{j}\to u\otimes w)\\
%= & \sum(b_{j}\to v)\otimes(b_{i}\to w)\otimes(a_{i}\to(a_{j}\to u)).
%\end{align*}
%Similarly one proves that $c_{U\otimes V,W}=(c_{U,P}\otimes\id_{V})(\id_{U}\otimes c_{V,W})$.

Now we prove the implication $\Longleftarrow$. So assume that $\lmod{H}$ is
braided and let $c$ be the braiding. Recall that $H$ is a left $H$-module with
$h\cdot k=hk$ for all $h,k\in H$.  Let  
\[
R=\tau_{H,H}(c_{H,H}(1\otimes1))=\sum a_{i}\otimes b_{i}.
\]
Since $C_{H,H}$ is invertible, $R$ is invertible. 

Let $U,V$ be two left $H$-modules and let $v\in V$ and $w\in W$. We consider
the maps $f_v:H\to V$, defined by $f_v(h)=h\cdot v$, and $f_w:H\to W$, defined
by $f_w(h)=h\cdot w$. By the naturality of $c$ we obtain:
\begin{equation}
\label{eq:QT_auxiliar}
c_{V,W}(v\otimes w)=\sum b_{i}\cdot w\otimes a_{i}\cdot v.
\end{equation}
In fact, 
\begin{align*} 
c_{V,W}(v\otimes w) & =c_{V,W}(f_v\otimes f_w)(1\otimes1)\\
 & =(f_w\otimes f_v)c_{H,H}(1\otimes1)\\
 & =(f_w\otimes f_v)\tau_{H,H}(R)\\
 & =\sum b_{i}\cdot w\otimes a_{i}\cdot v.
\end{align*}
Since $c_{V,W}$ is a morphism of left $H$-modules, 
\[
c_{H,H}(h_1\otimes h_2)=c_{H,H}(h\cdot(1\otimes1))=h\cdot c_{H,H}(1\otimes1)=\Delta(h)c_{H,H}(1\otimes1).
\]
Therefore, using \eqref{eq:QT_auxiliar} we obtain 
\begin{align*}
\Delta^{\mathrm{cop}}(h)R &= \tau_{H,H}(\Delta(h)c_{H,H}(1\otimes1))\\
&=\tau_{H,H}(c_{H,H}(h_1\otimes h_2))=\sum a_ih_1\otimes b_ih_2=R\Delta(h)
\end{align*}
for all $h\in H$. 

Now using \eqref{eq:QT_auxiliar} and the equation $c_{U,V\otimes
W}=(\id_{V}\otimes c_{U,W})(c_{U,V}\otimes\id_{W})$ we will
obtain $(\id\otimes\Delta)(R)=R_{13}R_{12}$. First we compute:
\begin{align*}
c_{H,H\otimes H}(1\otimes1\otimes1) & =(\id_{H}\otimes c_{H,H})(c_{H,H}\otimes\id_{H})(1\otimes1\otimes1)\\
 & =(\id_{H}\otimes c_{H,H})(c_{H,H}(1\otimes1)\otimes1)\\
 & =(\id_{H}\otimes c_{H,H})(\tau(R)\otimes1)\\
 & =\sum(\id_{H}\otimes c_{H,H})(b_{i}\otimes a_{i}\otimes1)\\
 & =\sum b_{i}\otimes c_{H,H}(a_{i}\otimes1)\\
 & =\sum b_{i}\otimes b_{j}\otimes a_{j}a_{i}.
\end{align*}
Using \eqref{eq:QT_auxiliar} with $V=H$ and $W=H\otimes H$ one obtains:
\[
c_{H,H\otimes H}(1\otimes1\otimes1)=\sum b_{i,1}\otimes b_{i,2}\otimes a_{i}
\]
and hence $(\id\otimes\Delta)(R)=R_{13}R_{12}$.  Similarly one proves
that $(\Delta\otimes\id)(R)=R_{12}R_{23}$. 
\end{proof}

\begin{exercise}\label{exercise:triangular}
Prove that a Hopf algebra $H$ is triangular if and only if $\lmod{H}$ is
symmetric. 
\end{exercise}

%\section{Algebras in categories}
Now there is a natural way of defining an algebra in a monoidal category. 

\begin{definition}
\index{algebras in monoidal categories}
Let $\mathcal{C}$ be a monoidal category. An \textbf{algebra} in
$\mathcal{C}$ is a triple $(A,m,u)$, where $A$ is an object of
$\mathcal{C}$, $m\in\hom(A\otimes A,A)$ and $u\in\hom(\I,A)$ such that 
\begin{gather*}
m(\id\otimes m)=m(m\otimes\id),\\
m(\id\otimes u)=\id=m(u\otimes\id).
\end{gather*}
Let $A$ and $B$ be algebras in $\mathcal{C}$ and $f\in\hom(A,B)$.
Then $f$ is a \textbf{morphism} (of algebras in $\mathcal{C}$)
if $m_{B}(f\otimes f)=fm_{A}$ and $fu_{A}=u_{B}$. This allows
us to define the category $\operatorname{Alg}(\mathcal{C})$
of algebras in $\mathcal{C}$. 
\end{definition}

\begin{example}
Let $\mathcal{C}=\mathrm{Vect}(\mathbb{K})$ be the category of
$\mathbb{K}$-vector spaces. An algebra $A$ in $\mathcal{C}$ is an algebra in
the usual sense.
\end{example}

\begin{example}
\index{module-algebra}
Let $\mathcal{C}=\lmod{H}$ be the category of left $H$-modules.  An algebra $A$
en $\mathcal{C}$ is an object of $\mathcal{C}$ such that $(a_{1}\to b)(a_{2}\to
b')=a\to bb'$ and $a\to1=\varepsilon(a)1$ for all $a,b\in A$. Hence an algebra
in $\lmod{H}$ is a left $H$-module-algebra.
\end{example}

\begin{example}
\index{comodule-algebra}
Let $\mathcal{C=}\lcomod{H}$ be the category of left $H$-comodules.  An algebra
$A$ in $\mathcal{C}$ is an object of $\mathcal{C}$ such that
$\delta(ab)=a_{-1}b{}_{-1}\otimes a_{0}b{}_{0}$ for all $a,b\in A$ and
$\delta(1)=1_{A}\otimes1_{H}$.  Hence an algebra in $\lcomod{H}$ is a left
$H$-comodule-algebra.
\end{example}

%\begin{exercise}
%Prove that 
%This is equivalent to ask for
%$\delta$ to be a morphism of algebras. 
%\end{exercise}

\begin{example}
\index{tensor product!of algebras in braided categories}
Let $(\mathcal{C},c)$ be a braided category and let $A$ and $B$ be two algebras
in $\mathcal{C}$. Then $A\otimes B$ is an algebra in $\mathcal{C}$ with
multiplication 
\[
m_{A\otimes B}=(m_A\otimes m_B)(\id_A\otimes c_{B,A}\otimes \id_B).
\]
\end{example}

%\section{Coalgebras in categories}
Similarly ones defines coalgebras in categories.

\begin{definition}
\index{coalgebras in monoidal categories}
Let $\mathcal{C}$ be a monoidal category. A \textbf{coalgebra}
$C$ in $\mathcal{C}$ is a triple $(C,\Delta,\varepsilon)$, where
$C$ is an object of $\mathcal{C}$, $\Delta\in\hom(C,C\otimes C)$
and $\varepsilon\in\hom(C,\mathbb{I})$ , and the following propositionerties
are satisfied: 
\begin{gather*}
(\Delta\otimes\id)\Delta=(\id\otimes\Delta)\Delta,\\
(\id\otimes\varepsilon)\Delta=(\varepsilon\otimes\id)\Delta=\id.
\end{gather*}
Let $C$ and $D$ be two coalgebras in $\mathcal{C}$ and $f\in\hom(C,D)$.
Then $f$ is a \textbf{morphism} (of coalgebras in $\mathcal{C}$)
if $\Delta_{D}f=(f\otimes f)\Delta_{C}$ and $\varepsilon_{D}f=\varepsilon_{C}$.
This allows us to define the category $\mathrm{Coalg}(\mathcal{C})$
of coalgebras in $\mathcal{C}$.
\end{definition}

\begin{example}
Let $\mathcal{C}=\mathrm{Vect}(\mathbb{K})$ be the category of $\mathbb{K}$-vector
spaces. A coalgebra $C$ in $\mathcal{C}$ is a coalgebra in the usual sense.
\end{example}

\begin{example}
\index{module-coalgebra}
A coalgebra $C$ in $\lmod{H}$ is an object of $\mathcal{C}$ such that 
\[
(h\to c)_{1}\otimes(h\to c)_{2}=h_{1}\to c_{1}\otimes h_{2}\to c_{2}
\]
and $\varepsilon(h\to c)=\varepsilon(h)\varepsilon(c)$ for all $h\in H$ and
$c\in C$. Hence a coalgebra in $\lmod{H}$ is a left $H$-module-coalgebra.
\end{example}

%\begin{exercise}
%This is equivalent to ask for the action $\to$ to be a morphism of
%coalgebras.
%\end{exercise}

\begin{example}
\index{comodule-coalgebra}
A coalgebra $C$ in $\lcomod{H}$ is an object of $\mathcal{C}$
such that 
\[
c_{1,-1}c_{2,-1}\otimes c_{1,0}\otimes c_{2,0}=c_{-1}\otimes c_{0,1}\otimes c_{0,2}
\]
and $c_{-1}\varepsilon_{C}(c_{0})=\varepsilon_{C}(c)1$ for all $c\in C$. Hence
a coalgebra in the category $\lcomod{H}$ is a left $H$-comodule-coalgebra.
\end{example}

\begin{example}
\index{tensor product!of coalgebras in braided categories}
Let $(\mathcal{C},c)$ be a braided category and let $C$ and $D$ be two
coalgebras in $\mathcal{C}$. Then $C\otimes D$ is an coalgebra in $\mathcal{C}$
with comultiplication 
\[
\Delta_{C\otimes D}=(\id_C\otimes c_{C,D}\otimes \id_D)(\Delta_C\otimes\Delta_D).
\]
\end{example}

%\section{Bialgebras and Hopf algebras in categories}
Now it is possible to define bialgebras and Hopf algebras in braided monoidal categories. 
%\section{Bialgebras and Hopf algebras in categories}

\begin{definition}
\index{Bialgebra!in a braided categories}
Let $\mathcal{C}$ be a braided monoidal category with braiding $c$.
A bialgebra in $\mathcal{C}$ is a tuple \textbf{$(B,m,\eta,\Delta,\varepsilon)$},
where $(B,m,\eta)$ is an algebra in $\mathcal{C}$, $(B,\Delta,\varepsilon)$
is a coalgebra in $\mathcal{C}$ and such that $\Delta\in\hom(B,B\otimes B)$
and $\varepsilon\in\hom(B,\mathbb{I})$ are morphism of algebras.
Here $B\otimes B$ is the algebra in $\mathcal{C}$ given by the product
\[
(m_{B}\otimes m_{B})(\id\otimes c_{B,B}\otimes\id).
\]
\end{definition}

\begin{exercise}
Let $H$ be a quasitriangular Hopf algebra with $R=\sum a_{i}\otimes b_{i}$.
Then $\lmod{H}$ is a braided monoidal category with braiding
\[
c_{V,W}(v\otimes w)=\sum_{i}b_{i}\cdot w\otimes a_{i}\cdot v.
\]
Prove that $H$ is a bialgebra in $\mathcal{C}$ if $H$ is an algebra
and a coalgebra in $\lmod{H}$ and 
\[
(hh')_{1}\otimes(hh')_{2}=\sum_{i}h_{1}(b_{i}\cdot h'_{1})\otimes(a_{i}\cdot h_{2})h'_{2}
\]
for all $h,h'\in H$.
\end{exercise}


\section{Radford biproduct}

Our goal is to know when it is possible to make $A\otimes H$ a bialgebra, where
the algebra structure is given by the smash product:
\[
(a\otimes h)(a'\otimes h')=a(h_{1}\to a')\otimes h_{2}h'
\]
for all $a,a'\in A$, $h,h'\in H$, and the coalgebra structure is
the smash coproduct:
\[
\Delta(a\otimes h)=(a_{1}\otimes a_{2,-1}h_{1})\otimes(a_{2,0}\otimes h_{2})
\]
for all $a\in A$, $h\in H$. This is the \textbf{Radford biproduct}.

\begin{theorem}[Radford]
\label{theorem:radford}
Let $H$ be Hopf algebra, and let $A$ be an algebra and a coalgebra such that
$(A,\rightarrow)$ a left $H$-module-algebra and $(A,\delta)$ a left
$H$-comodule-coalgebra.  Assume that 
\begin{gather}
A\text{ is a left \ensuremath{H}-comodule-algebra},\label{eq:radford_1}\\
A\text{ is a left \ensuremath{H}-module-coalgebra,}\label{eq:radford_2}\\
\varepsilon_{A}\text{ is a morphism of algebras,}\label{eq:radford_3}\\
\Delta(1_{A})=1_{A}\otimes1_{A},\label{eq:radford_4}\\
\Delta(aa')=a_{1}\left(a_{2,-1}\rightarrow a'_{1}\right)\otimes a_{2,0}a'_{2},\label{eq:radford_5}\\
\left(h_{1}\rightarrow a\right)_{-1}h_{2}\otimes\left(h_{1}\rightarrow a\right)_{0}=h_{1}a_{-1}\otimes h_{2}\rightarrow a_{0}.\label{eq:radford_6}
\end{gather}
for all $a,a'\in A$, $h\in H$.
Then the vector space $A\otimes H$ is a bialgebra with the algebra structure
given by the left smash product and the coalgebra is the left smash coproduct.
Furthermore, if $A$ has an antipode $S_A$, then $A\otimes H$ is a Hopf algebra
with antipode
\[
S(a\otimes h)=(1\otimes S_{H}(a_{-1}h))(S_{A}(a_{0})\otimes1)
\]
for all $a\in A$, $h\in H$.
\end{theorem}

\begin{proof}
We first prove that $\varepsilon$ is a morphism of algebras: 
\begin{align*}
\varepsilon((a\otimes h)(a'\otimes h')) & =\varepsilon(a(h_{1}\to a')\otimes h_{2}h')\\
 & =\varepsilon(a(h_{1}\to a')\varepsilon(h_{2}h')\\
 & =\varepsilon(a)\varepsilon(h_{1}\to a')\varepsilon(h_{2})\varepsilon(h')\\
 & =\varepsilon(a)\varepsilon(h_{1})\varepsilon(a')\varepsilon(h_{2})\varepsilon(h')\\
 & =\varepsilon(a)\varepsilon(h)\varepsilon(a')\varepsilon(h')\\
 & =\varepsilon(a\otimes h)\varepsilon(a'\otimes h'),
 \end{align*}
and $\varepsilon(1\otimes1)=1$.  Now we prove that $\Delta$ is a morphism of
algebras. By \eqref{eq:radford_4}, we need to prove that $\Delta$ is
multiplicative. We compute:
\begin{align*}
\Delta & (a\otimes h)\Delta(a'\otimes h')\\
 & =(a_{1}\otimes a_{2,-1}h_{1}\otimes a_{2,0}\otimes h_{2})(a'_{1}\otimes a'_{2,-1}h'_{1}\otimes a'_{2,0}\otimes h'_{2})\\
 & =(a_{1}\otimes a_{2,-1}h_{1})(a'_{1}\otimes a'_{2,-1})\otimes(a_{2,0}\otimes h_{2})(a'_{2,0}\otimes h'_{2})\\
 & =a_{1}((a_{2,-1}h_{1})_{1}\to a'_{1})\otimes(a_{2,-1}h_{1})_{2}a'_{2,-1}h'_{1}\otimes a_{2,0}(h_{2,1}\to a'_{2,0})\otimes h_{2,2}h'_{2}\\
 & =a_{1}((a_{2,-1,1}h_{1,1})\to a'_{1})\otimes a_{2,-1,2}h_{1,2}a'_{2,-1}h'_{1}\otimes a_{2,0}(h_{2,1}\to a'_{2,0})\otimes h_{2,2}h'_{2}\\
 & =a_{1}((a_{2,-1,1}h_{1})\to a'_{1})\otimes a_{2,-1,2}h_{2}a'_{2,-1}h'_{1}\otimes a_{2,0}(h_{3}\to a'_{2,0})\otimes h_{4}h'_{2}.\end{align*}
On the other hand, we compute:
\begin{align*}
\Delta & ((a\otimes h)(a'\otimes h'))\\
 & =\Delta(a(h_{1}\to a')\otimes h_{2}h')\\
 & =(a(h_{1}\to a'))_{1}\otimes(a(h_{1}\to a'))_{2,-1}(h_{2}h')_{1}\otimes(a(h_{1}\to a'))_{2,0}\otimes(h_{2}h')_{2}\\
 & =(a(h_{1}\to a'))_{1}\otimes(a(h_{1}\to a'))_{2,-1}h_{2}h'_{1}\otimes(a(h_{1}\to a'))_{2,0}\otimes h_{3}h'_{2}\\
 & =a_{1}(a_{2,-1}\to(h_{1}\to a')_{1})\otimes(a_{2,0}(h_{1}\to a')_{2})_{-1}h_{2}h'_{1}\otimes(a_{2,0}(h_{1}\to a')_{2})_{0}\otimes h_{2}h'_{2}\\
 & =a_{1}(a_{2,-1}\to(h_{1}\to a'_{1})\otimes(a_{2,0}(h_{2}\to a'_{2}))_{-1}h_{3}h'_{1}\otimes(a_{2,0}(h_{2}\to a'_{2}))_{0}\otimes h_{4}h'_{2}\\
 & =a_{1}(a_{2,-1}h_{1}\to a'_{1})\otimes a_{2,0,-1}(h_{2}\to a'_{2})_{-1}h_{3}h'_{1}\otimes a_{2,0,0}(h_{2}\to a'_{2})_{0}\otimes h_{4}h'_{2}\\
 & =a_{1}(a_{2,-1}h_{1}\to a'_{1})\otimes a_{2,0,-1}(h_{2}a'_{2,-1})h'_{1}\otimes a_{2,0,0}(h_{3}\to a'_{2,0})\otimes h_{4}h'_{2}\\
 & =a_{1}(a_{2,-1,1}h_{1}\to a'_{1})\otimes a_{2,-1,2}h_{2}a'_{2,-1}h'_{1}\otimes a_{2,0}(h_{3}\to a'_{2,0})\otimes h_{4}h'_{2}.
\end{align*}
Since $A$ is a left $H$-comodule-coalgebra and $a_{1,-1}a_{2,-1}\otimes a_{1,0}\otimes a_{2,0}=a_{-1}\otimes a_{0,1}\otimes a_{0,2}$,
we obtain: 
\begin{align*}
S((a\otimes h)_{1})(a\otimes h)_{2} & =S(a_{1}\otimes a_{2,-1}h_{1})(a_{2,0}\otimes h_{2})\\
 & =(1\otimes S_{H}(a_{1,-1}a_{2,-1}h_{1}))(S_{A}(a_{1,0})\otimes1)(a_{2,0}\otimes h_{2})\\
 & =S_{A}(a_{1,0})a_{2,0}\otimes S_{H}(a_{1,-1}a_{2,-1}h_{1})h_{2}\\
 & =\varepsilon(a_{0})1\otimes S_{H}(a_{-1}h_{1})h_{2}\\
 & =\varepsilon(a)\varepsilon(h)1\otimes1.
\end{align*}
Since $a_{-1}\otimes a_{0,-1}\otimes a_{0,0}=a_{-1,1}\otimes a_{-1,2}\otimes a_{0}$ we obtain:
\begin{align*}
(a\otimes h)_{1}S((a\otimes h)_{2}) & =(a_{1}\otimes a_{2,-1}h_{1})S(a_{2,0}\otimes h_{2})\\
 & =(a_{1}\otimes a_{2,-1}h_{1})(1\otimes S_{H}(a_{2,0,-1}h_{2}))(S_{A}(a_{2,0,0})\otimes1)\\
 & =a_{1}S_{A}(a_{2,0,0})\otimes a_{2,-1}h_{1}S_{H}(a_{2,0,-1}h_{2})\\
 & =a_{1}S_{A}(a_{2,0,0})\otimes a_{2,-1}h_{1}S_{H}(h_{2})S_{H}(a_{2,0,-1})\\
 & =a_{1}S_{A}(a_{2,0})\otimes a_{2,-1,1}S_{H}(a_{2,-1,2})\varepsilon(h)\\
 & =a_{1}S_{A}(a_{2})\otimes1\varepsilon(h)\\
 & =1\otimes1\varepsilon(a)\varepsilon(h).\qedhere
 \end{align*}
\end{proof}

\begin{exercise}
Prove that the Radford biproduct over $A\otimes H$ is commutative if and only
if $A$ and $H$ are commutatives and the action $\to$ is trivial.  Similarly,
the Radford biproduct over $A\otimes H$ is cocommutative if and only if $A$ and
$H$ are cocommutative and the coaction $\delta_{A}$ is trivial.
\end{exercise}

\begin{exercise}
Prove the converse of Theorem \ref{theorem:radford}: assume that $H$ is a
bialgebra, $A$ is a left $H$-module-algebra and a left $H$-comodule coalgebra
and the Radford biproduct $A\otimes H$ is a bialgebra. Then
\eqref{eq:radford_1}--\eqref{eq:radford_6} are satisfied.
\end{exercise}

%\begin{exercise}
%Let $(H,R)$ be a quasitriangular Hopf algebra, and let $B$ be a bialgebra in
%the category $\H_\mathcal{M}$. Prove that $B$ is a left $H$-comodule-algebra via
%$\delta(b)=R^{-1}(1\otimes b)$ and $B\otimes H$ with the Radford biproduct is a
%bialgebra.  Furthermore, if $B$ is a triangular algebra then the biproduct is
%quasitriangular.
%\end{exercise}

Similarly, it is possible to put on $H\otimes B$ a bialgebra structure, where
the algebra structure is given by the smash product over $H\otimes B$ and the
coalgebra is given by the smash coproduct over $H\otimes B$. For that purpose
we need $B$ to be a right $H$-module-algebra and a right
$H$-comodule-coalgebra. In this case, the necessary and sufficient conditions
are:
\begin{gather*}
B\text{ is a right \ensuremath{H}-comodule-algebra,}\\
B\text{ a right \ensuremath{H}-module-coalgebra,}\\
\varepsilon_{B}\text{ is a morphism of algebras,}\\
\Delta(1_{B})=1_{B}\otimes1_{B},\\
\Delta(bb')=b_{1}b'_{1,0}\otimes(b_{2}\leftarrow b'_{1,1})b'_{2},\\
(b_{0}\leftarrow h_{1})\otimes b_{1}h_{2}=(b\leftarrow h_{2})_{0}\otimes h_{1}(b\leftarrow h_{2})_{1}.
\end{gather*}

A different and important bialgebra structure on $A\otimes H$ is
the so-called \textbf{Majid product}. Let $A$ be an left $H$-module-algebra
and $H$ be a right $A$-comodule-coalgebra. On the vector space $A\otimes H$
we consider the algebra structure given by the smash product on $A\otimes H$ and the coalgebra
structure given on $A\otimes H$, i.e.,
\begin{align*}
(a\otimes h)(a'\otimes h') & =a(h_{1}\rightarrow a')\otimes h_{2}h',\\
\Delta(a\otimes h) & =a_{1}\otimes h_{1,0}\otimes a_{2}h_{1,1}\otimes h_{2}.\end{align*}
Then $A\otimes H$ is a bialgebra if and only if \begin{gather*}
\varepsilon(h\rightarrow a)=\varepsilon_{H}(h)\varepsilon_{A}(a),\\
\Delta(h\rightarrow a)=h_{1,0}\rightarrow a_{1}\otimes h_{1,1}(h_{2}\rightarrow a_{2}),\\
\delta(1)=1\otimes1,\\
\delta(hh')=h_{1,0}h'_{0}\otimes h_{1,1}(h_{2}\rightarrow h'_{1}),\\
h_{2,0}\otimes(h_{1}\rightarrow a)h_{2,1}=h_{1,0}\otimes h_{1,1}(h_{2}\rightarrow a).
\end{gather*}


%\section{Radford bosonization}
The following result is known as the Radford's bosonization. 

\begin{theorem}[Radford]
\label{theorem:bosonization}
Let $H$ be a Hopf algebra with bijective antipode. There exists a bijective
correspondence between
\begin{enumerate}
\item Hopf algebras $A$ with morphisms $H\xrightarrow{i}A\xrightarrow{p}H$
such that $pi=\textrm{id}_{H}$.
\item Hopf algebras in the category $_{H}^{H}\mathcal{YD}$.
\end{enumerate}
\end{theorem}

\begin{proof}
Assume (1). We claim that 
\[
R=A^{\textrm{co}H}=\{a\in A\mid(\textrm{id}\otimes p)\Delta(a)=a\otimes1\}
\]
is a Hopf algebra in the category of left Yetter-Drinfeld modules.
It is clear that $R$ is a subalgebra of $A$. Now define
\begin{align*}
\Delta_{R}(r) & =r_{1}iSp(r_{2})\otimes r_{3},\\
S_{R}(r) & =ip(r_{1})S(r_{2}),\\
h\to r & =i(h_{1})riS(h_{2}),\\
\delta(r) & =(p\otimes\textrm{id})\Delta(r),
\end{align*}
for all $r\in R$ and $h\in H$. 
We write $\Delta_{R}(r)=r^{1}\otimes r^{2}$ to distinguish $\Delta_{R}(r)$
and $\Delta_{A}(r)=r_{1}\otimes r_{2}$. We claim that $\Delta_{R}$
is coassociative:
\begin{align*}
(\textrm{id}\otimes\Delta_{R})\Delta_{R}(r) & =(\textrm{id}\otimes\Delta_{R})(r_{1}iSp(r_{2})\otimes r_{3})\\
 & =r_{1}iSp(r_{2})\otimes r_{3,1}iSp(r_{3,2})\otimes r_{3,3}\\
 & =r_{1}iSp(r_{2})\otimes r_{3}iSp(r_{4})\otimes r_{5}.
 \end{align*}
On the other hand:
\begin{align*}
(\Delta_{R}\otimes\textrm{id})\Delta_{R}(r) & =(\Delta_{R}\otimes\textrm{id})(r_{1}iSp(r_{2})\otimes r_{3})\\
 & =\Delta_{R}(r_{1}iSp(r_{2}))\otimes r_{3}\\
 & =[r_{1}iSp(r_{2})]_{1}iSp([r_{1}iSp(r_{2})]_{2})\otimes[r_{1}iSp(r_{2})]_{3}\otimes r_{3}\\
 & =r_{1,1}[iSp(r_{2})]_{1}iSp(r_{1,2}[iS_{H}p(r_{2})]_{2})\otimes r_{1,3}[iSp(r_{2})]_{3}\otimes r_{3}\\
 & =r_{1}iSp(r_{6})iSp(r_{2}iSp(r_{5}))\otimes r_{3}iSp(r_{4})\otimes r_{7}\\
 & =r_{1}iS[p(r_{2})Sp(r_{5})r_{6}]\otimes r_{3}iSp(r_{4})\otimes r_{7}\\
 & =r_{1}iSp(r_{2})\otimes r_{3}iSp(r_{4})\otimes r_{5}.\end{align*}
Hence $R$ is an algebra and a coalgebra. 

We claim that $R$ is a left $H$-comodule-algebra, since \[
\delta(1)=(p\otimes\textrm{id})\Delta(1)=p(1)\otimes1=1\otimes1,\]
and \[
\delta(rr')=p(r_{1}r_{1}')\otimes r_{2}r'_{2}=p(r_{1})p(r'_{1})\otimes r_{2}r'_{2}=r_{-1}r'_{-1}\otimes r_{0}r'_{0}.\]

We claim that $R$ if a left $H$-comodule-coalgebra, since \[
r_{-1}\varepsilon(r_{0})=p(r_{1})\varepsilon(r_{2})=p(r_{1}\varepsilon(r_{2}))=p(r)\]
and since $r\in R$, \[
\varepsilon(r)=(\varepsilon\otimes\textrm{id})(r\otimes1)=(\varepsilon\otimes\textrm{id})(\textrm{id}\otimes p)\Delta(r)=\varepsilon(r_{1})p(r_{2})=p(r).\]
Futhermore, \begin{align*}
(r^{1})_{-1}(r^{2})_{-1}\otimes(r^{1})_{0}\otimes(r^{2})_{0} & =p[(r_{1}iSpr_{2})_{1}r_{3,1}]\otimes(r_{1}iSp(r_{2}))_{2}\otimes r_{3,2}\\
 & =p[r_{1,1}i(Spr_{2})_{1}r_{3,1}]\otimes r_{1,2}i(Spr_{2})_{2}\otimes r_{3,2}\\
 & =p(r_{1}iSpr_{4}r_{5})\otimes r_{2}iSp(r_{3})\otimes r_{6}\\
 & =p(r_{1})\otimes r_{2}iSpr_{3}\otimes r_{4}.\end{align*}
and on the other hand,\begin{align*}
r_{-1}\otimes(r_{0})^{1}\otimes(r_{0})^{2} & =r_{-1}\otimes\Delta_{R}(r_{0})\\
 & =p(r_{1})\otimes\Delta_{R}(r_{2})\\
 & =p(r_{1})\otimes r_{2}iSp(r_{3})\otimes r_{4}.
\end{align*}
We claim that $R$ is a left $H$-module-algebra, since 
\[
h\to1=ih_{1}iSh_{2}=i(h_{1}Sh_{2})=\varepsilon(h)i(1)=\varepsilon(h)1
\]
and 
\begin{align*}
(h_{1}\to r)(h_{2}\to r') & =ih_{1,1}riSh_{1,2}ih_{2,1}r'iSh_{2,2}\\
 & =ih_{1}riSh_{2}ih_{3}r'iSh_{4}\\
 & =ih_{1}r\varepsilon(h_{2})r'iSh_{3}\\
 & =ih_{1}rr'iSh_{2}\\
 & =h\to(rr').
\end{align*}
We claim that $R$ is a left $H$-module-coalgebra, since \[
\varepsilon(h\to r)=\varepsilon(ih_{1}aiSh_{2})=\varepsilon(ih_{1})\varepsilon(r)\varepsilon(iSh_{2})=\varepsilon(h)\varepsilon(r)\]
and \begin{align*}
\Delta_{R}(h\to r) & =\Delta_{R}(ih_{1}riSh_{2})\\
 & =[ih_{1}riSh_{2}]_{1}iSp([ih_{1}riSh_{2}]_{2})\otimes[ih_{1}riSh_{2}]_{3}\\
 & =ih_{1,1}r_{1}iS(h_{2})_{2}iSp(ih_{1,2}r_{2}iS(h_{2})_{2}\otimes ih_{1,3}r_{3}iS(h_{2})_{3}\\
 & =ih_{1}r_{1}iSh_{6}iSp[ih_{2}r_{2}iSh_{5}]\otimes ih_{3}r_{3}iSh_{4}\\
 & =ih_{1}r_{1}iSh_{6}iS[h_{2}pr_{2}Sh_{5}]\otimes ih_{3}r_{3}iSh_{4}\\
 & =ih_{1}r_{1}iS[h_{2}pr_{2}Sh_{5}h_{6}]\otimes ih_{3}r_{3}iSh_{4}\\
 & =ih_{1}r_{1}iS(h_{2}pr_{2})\varepsilon(h_{5})\otimes ih_{3}r_{3}iSh_{4}\\
 & =ih_{1}r_{1}iSpr_{2}Sh_{2}\otimes ih_{3}r_{3}iSh_{4}\end{align*}
and 
\begin{align*}
h_{1}\to r^{1}\otimes h_{2}\to r^{2} & = h_{1}\to r_{1}iSpr_{2}\otimes h_{2}\to r_{3}\\
 & = ih_{1,1}r_{1}iSpr_{2}iSh_{1,2}\otimes ih_{2,1}r_{3}iSh_{2,2}\\
 & = ih_{1}r_{1}iSpr_{2}iSh_{2}\otimes ih_{3}r_{3}iSh_{4}
\end{align*}
To prove that $R$ is a bialgebra in $_{H}^{H}\mathcal{YD}$ it remains
to prove that $\Delta_{R}$ is a morphism in $_{H}^{H}\mathcal{YD}$.
We compute: 
\begin{align*}
\Delta_{R}(rr') & = (rr')_{1}iSp((rr')_{2})\otimes(rr')_{3}\\
 & = r_{1}r'_{1}iSp(r_{2}r'_{2})\otimes r_{3}r'_{3}\\
 & = r_{1}r'_{1}iSp(r'_{2})iSp(r_{2})\otimes r_{3}r'_{3}.
\end{align*}
On the other hand,
\begin{align*}
r^{1}((r^{2})_{-1}\to r'^{1})\otimes(r^{2})_{0}r'^{2} &=r_{1}iSp(r_{2})(r_{3,-1}\to(r'_{1}iSp(r'_{2}))\otimes r_{3,0}r'_{3}\\
 &=r_{1}iSp(r_{2})(p(r_{3,1})\to r'_{1}iSp(r'_{2}))\otimes r_{3,2}r'_{3}\\
 &=r_{1}iSp(r_{2})i(p(r_{3})_{1})r'_{1}iSp(r'_{2})iS(p(r_{2})_{2})\otimes r_{4}r'_{3}\\
 &=r_{1}i[Sp(r_{2})p(r_{3})]r'_{1}iS[p(r'_{2})iSp(r_{4})]\otimes r_{5}r'_{3}\\
 &=r_{1}\varepsilon(r_{2})r'_{1}iSp(r'_{2})iSp(r_{3})\otimes r_{4}r'_{3}\\
 &=r_{1}r'_{1}iSp(r'_{2})iSp(r_{2})\otimes r_{3}r'_{3}.
\end{align*}

Conversely, let $R$ be a Hopf algebra in the category of Yetter-Drinfeld
modules. Then the Radford biproduct $R\otimes H$ is a Hopf algebra by Theorem
\ref{theorem:radford}. The maps $p:R\otimes H\to H$, defined by $r\otimes
h\mapsto\varepsilon(r)h$, and $i:H\to R\otimes H$, defined by $h\mapsto
1\otimes h$ are Hopf algebra morphisms and $p\circ i=\id$. 
%Furthermore,
%\[
%R\otimes1=\{a\in R\otimes H\mid (\id\otimes p)\Delta(a)=a\otimes1\}.
%\]
%Now the claim follows from the following exercise.
\end{proof}

\begin{exercise}
Let $A$ and $H$ be two Hopf algebras such that there exist Hopf algebras
morphisms $H\xrightarrow{i}A\xrightarrow{p}H$ such that $pi=\id_H$. Let
$R=A^{\mathrm{co}H}$ and consider the map $\omega:A\to R$ defined by $a\mapsto
a_1ip(Sa_2)$. 
\begin{enumerate}
    \item Prove that the maps $\alpha:A\to R\otimes H$, $\alpha(a)=\omega(a_1)\otimes p(a_2)$, and 
    $\beta:R\otimes H\to A$, $r\otimes h\mapsto ri(h)$     
    are Hopf algebra homomorphisms. 
    \item Prove that $\alpha\circ\beta=\id_{R\otimes H}$ and $\beta\circ\alpha=\id_A$ and 
    conclude that $A\simeq R\otimes H$ as Hopf algebras.
\end{enumerate}
\end{exercise}

\section{Some solutions}

%\subsection*{Quasitriangular Hopf algebras}

\begin{sol}{exercise:QT_hexagons}
We prove \eqref{eq:(Rx1)(1xR)}. A straightforward computation shows
that 
\[
R_{U\otimes V,W}(u\otimes v\otimes w)=\sum b_{i}\cdot w\otimes a_{i1}\cdot u\otimes a_{i2}\cdot v.
\]
On the other hand,
\begin{align*}
(R_{U,W}\otimes\id_{V}) & (\id_{U}\otimes R_{V,W})(u\otimes v\otimes w)\\
 & =\sum(R_{U,W}\otimes\id_{V})(u\otimes b_{i}\cdot w\otimes a_{i}\cdot v)\\
 & =\sum(b_{j}b_{i})\cdot w\otimes a_{j}\cdot w\otimes a_{i}\cdot w
\end{align*}
and the claim follows from Equation \eqref{QT:2}. The proof for \eqref{eq:(1xR)(Rx1)}
is similar. 
\end{sol}

\begin{sol}{exercise:VW=WV}
Define $\phi:V\otimes W\to W\otimes V$ by $v\otimes w\mapsto R^{-1}\cdot (w\otimes
v)$. Then $\phi$ is an isomorphism of left $H$-modules:
\begin{align*}
\phi(h\cdot (v\otimes w))&=R^{-1}(h_2\cdot w\otimes h_1\cdot v)\\
&=R^{-1}\tau\Delta(h)(w\otimes v)=\Delta(h)R^{-1}(w\otimes v)=h\cdot \phi(v\otimes w).
\end{align*}
\end{sol}

%\subsection*{Actions and coactions}

\begin{sol}{exercise:adjoint}
First we prove that $(H,\rightarrow)$ is a left $H$-module.  We
compute 
\begin{align*} 
b\rightarrow(a\rightarrow x) & =b\rightarrow(a_{1}xS(a_{2}))\\
 & =b_{1}a_{1}xS(a_{2})S(b_{2})\\
 & =(ba)_{1}xS\left((ba)_{2}\right)\\
 & =(ba)\rightarrow x.
\end{align*}
Then $(H,\rightarrow)$ is a left $H$-module, since it is trivial
to prove that $1\rightarrow x=x$. To prove that $(H,\rightarrow)$
is a left module-algebra over $H$ we compute:
\[
a\rightarrow=a_{1}1S(a_{2})=\varepsilon(a)1,
\]
and
\begin{align*}
(a_{1}\rightarrow x)(a_{2}\rightarrow y) & =(a_{1,1}xS(a_{1,2}))(a_{2,1}yS(a_{2,2}))\\
 & =a_{1}x\varepsilon(a_{2})yS(a_{3})\\
 & =a_{1}xyS(a_{2})\\
 & =a\rightarrow(xy).
\end{align*}
The proof for the right adjoint action is similar. 
\end{sol}

\begin{sol}{exercise:left_smash}
We first prove that $1\otimes1$ is the unit: 
\begin{align*}
(1\otimes1)(a\otimes h) & =1(1\rightarrow a)\otimes1h=1a\otimes h=a\otimes h,\\
(a\otimes h)(1\otimes1) & =a(h_{1}\rightarrow1)\otimes h_{2}1=a(\varepsilon(h_{1})1)\otimes h_{2}=a\otimes h.
\end{align*}
Now we prove the associativity. A direct computation shows that 
\begin{align*}
\left((a\otimes h)(b\otimes g)\right)(c\otimes k) & =(a(h_{1}\rightarrow b)\otimes h_{2}g)(c\otimes k)\\
 & =(a(h_{1}\rightarrow b))((h_{2}g)_{1}\rightarrow c)\otimes(h_{2}g)_{2}k\\
 & =a(h_{1}\rightarrow b)(h_{2}g_{1}\rightarrow c)\otimes(h_{3}g_{2})k.
\end{align*}
On the other hand, since $A$ is an $H$-module-algebra,
\begin{align*}
(a\otimes h)\left((b\otimes g)(c\otimes k)\right) & =(a\otimes h)(b(g_{1}\rightarrow c)\otimes g_{2}k)\\
 & =a(h_{1}\rightarrow(b(g_{1}\rightarrow c)))\otimes h_{2}(g_{2}k)\\
 & =a(h_{1}\rightarrow b)(h_{2}\rightarrow(g_{1}\rightarrow c))\otimes h_{3}(g_{2}k).
\end{align*}
\end{sol}

\begin{sol}{exercise:smash_coleft}
We first prove that $\varepsilon$ is the counit:
\begin{align*}
(\varepsilon\otimes\textrm{id})\Delta(c\otimes h) & =(\varepsilon\otimes\textrm{id})(c_{1}\otimes c_{2,-1}h_{1}\otimes c_{2,0}\otimes h_{2})\\
 & =\varepsilon(c_{1}\otimes c_{2,-1}h_{1})c_{2,0}\times h_{2}\\
 & =\varepsilon_{C}(c_{1})\varepsilon_{H}(c_{2,-1}h_{1})c_{2,0}\otimes h_{2}\\
 & =\varepsilon_{C}(c_{1})\varepsilon_{H}(h_{1})(c_{2,-1})c_{2,0}\otimes\varepsilon_{H}(h_{1})h_{2}\\
 & =c\otimes h,
\end{align*}
where the last equality holds since $(\varepsilon_{H}\otimes\textrm{id})\delta=\textrm{id}$
and hence 
\[
c=(\varepsilon_{H}\otimes\textrm{id})\delta(c)=(\varepsilon_{H}\otimes\textrm{id})\delta(\varepsilon_{C}(c_{1})c_{2})=\varepsilon_{C}(c_{1})\varepsilon_{H}(c_{2,-1})c_{2,0}.
\]
Similarly we obtain that $\textrm{(id}\otimes\varepsilon)\Delta=\textrm{id}$.
Now we prove the coassociativity: 
\begin{align*}
(\Delta\otimes\textrm{id})\Delta(c\otimes h) & =(\Delta\otimes\textrm{id})((c_{1}\otimes c_{2,-1}h_{1})\otimes(c_{2,0}\otimes h_{2}))\\
 & =\Delta(c_{1}\otimes c_{2,-1}h_{1})\otimes(c_{2,0}\otimes h_{2})\\
 & =c_{1,1}\otimes c_{1,2,-1}(c_{2,-1}h_{1})_{1}\otimes c_{1,2,0}\otimes(c_{2,-1}h_{1})_{2}\otimes c_{2,0}\otimes h_{2}\\
 & =c_{1}\otimes c_{2,-1}(c_{3,-1}h_{1})_{1}\otimes c_{2,0}\otimes(c_{3,-1}h_{1})_{2}\otimes c_{3,0}\otimes h_{2}\\
 & =c_{1}\otimes c_{2,-1}c_{3,-1,1}h_{1}\otimes c_{2,0}\otimes c_{3,-1,2}h_{2}\otimes c_{3,0}\otimes h_{3}\\
 & =c_{1}\otimes c_{2,-1}c_{3,-1}h_{1}\otimes c_{2,0}\otimes c_{3,0,-1}h_{2}\otimes c_{3,0,0}\otimes h_{3}\\
 & =c_{1}\otimes c_{2,-1}c_{3,-2}h_{1}\otimes c_{2,0}\otimes c_{3,-1}h_{2}\otimes c_{3,0}\otimes h_{3},
\end{align*}
where we have used that $C$ is a left $H$-comodule-coalgebra: 
\[
c_{-1,1}\otimes c_{-1,2}\otimes c_{0}=c_{-1}\otimes c_{0,-1}\otimes c_{0,0}=c_{-2}\otimes c_{-1}\otimes c_{0}\in H\otimes H\otimes C.
\]
On the other hand, 
\begin{align*}
(\textrm{id}\otimes\Delta)\Delta(c\otimes h) & =(\textrm{id}\otimes\Delta)((c_{1}\otimes c_{2,-1}h_{1})\otimes(c_{2,0}\otimes h_{2}))\\
 & =c_{1}\otimes c_{2,-1}h_{1}\otimes\Delta(c_{2,0}\otimes h_{2})\\
 & =c_{1}\otimes c_{2,-1}h_{1}\otimes c_{2,0,1}\otimes c_{2,0,2,-1}h_{2,1}\otimes c_{2,0,2,0}\otimes h_{2,2}\\
 & =c_{1}\otimes c_{2,-1}h_{1}\otimes c_{2,0,1}\otimes c_{2,0,2,-1}h_{2}\otimes c_{2,0,2,0}\otimes h_{3}\\
 & =c_{1}\otimes c_{2,-1}c_{3,-1}h_{1}\otimes c_{2,0}\otimes c_{3,0,-1}h_{2}\otimes c_{3,0,0}\otimes h_{3}\\
 & =c_{1}\otimes c_{2,-1}c_{3,-2}h_{1}\otimes c_{2,0}\otimes c_{3,-1}h_{2}\otimes c_{3,0}\otimes h_{3},
\end{align*}
where we have used that $c_{-1}\otimes c_{0,1}\otimes c_{0,2}=c_{1,-1}c_{2,-1}\otimes c_{1,0}\otimes c_{2,0}$
since $C$ is a left $H$-comodule-coalgebra.
\end{sol}

\begin{sol}{exercise:YD_condition}
Assume that \eqref{eq:YD} holds. Then 
\[
\delta(h_{1}\to v)=(h_{1}\to v)_{-1}\otimes(h_{1}\to v)_{0}=h_{1,1}v_{-1}Sh_{1,3}\otimes h_{1,2}\to v_{0}.
\]
Hence 
\[
(h_{1}\to v)_{-1}h_{2}\otimes(h_{1}\to v)_{0}=h_{1,1}v_{-1}Sh_{1,3}h_{2}\otimes h_{1,2}\to v_{0}=h_{1}v_{-1}\otimes h_{2}\to v_{0}.
\]
Conversely, assume that \eqref{eq:left_left_YD_equivalent} holds. Then
\begin{align*}
	(m\otimes&\id)(h_{11}v_{-1}\otimes Sh_2\otimes(h_{12}\rightarrow v_0) )\\  
	&=(m\otimes\id)\left( (h_{11}\rightarrow v)_{-1}h_{12}\otimes Sh_2\otimes (h_{11}\rightarrow v)_0 \right)\\
	&=(h_1\rightarrow v)_{-1}h_2Sh_3\otimes (h_1\rightarrow v)_0\\
	&=(h\rightarrow v)_{-1}\otimes (h\rightarrow v)_0.
\end{align*}
\end{sol}

\begin{sol}{exercise:YD_tensor}
To prove the compatibility condition \eqref{eq:YD} we compute
\begin{align*}
\delta(h\rightarrow(v\otimes w)) & =\delta(h_{1}\rightarrow v\otimes h_{2}\rightarrow w)\\
 & =(h_{1}\rightarrow v)_{-1}(h_{2}\rightarrow w)_{-1}\otimes(h_{1}\rightarrow v)_{0}\otimes(h_{2}\rightarrow w)_{0}\\
 & =(h_{1}v_{-1}(Sh_{3})h_{4}w_{-1}Sh_{6}\otimes(h_{2}\rightarrow v_{0})\otimes(h_{5}\rightarrow w_{0})\\
 & =h_{1}v_{-1}w_{-1}Sh_{4}\otimes(h_{2}\rightarrow v_{0})\otimes(h_{3}\rightarrow w_{0})\\
 & =h_{1}v_{-1}w_{-1}Sh_{3}\otimes h_{2}\rightarrow(v_{0}\otimes w_{0})\\
 & =h_{1}(v\otimes w)_{-1}Sh_{3}\otimes h_{2}\rightarrow(v\otimes w)_{0}.
\end{align*}
\end{sol}


\begin{sol}{exercise:triangular}
Assume first that $H$ es triangular. Then $\tau(R)=R$ and hence
$c_{V,W}c_{W,V}=\id_{V\otimes W}$. Conversely, using 
\eqref{eq:QT_auxiliar} we obtain 
\[
1\otimes1=c_{H,H}(c_{H,H}(1\otimes1))=c_{H,H}(\tau(R))=\tau(R)R.
\]
\end{sol}

\bibliographystyle{plain}
\bibliography{refs}

%\printindex

\end{document}
