%\section{Actions and coactions}

\section{(Co)actions on (co)algebras}

\begin{definition}
\label{def:module_algebra}
\index{module-algebra}
Let $H$ be a Hopf algebra. A left \textbf{$H$-module-algebra} is an algebra
$A$ with a left $H$-module structure such that
\begin{align*}
&h\rightarrow(ab)=(h_{1}\rightarrow a)(h_{2}\rightarrow b),\\
&h\rightarrow1=\varepsilon(h)1
\end{align*}
for all $h\in H$ and $a,b\in A$. 
\end{definition}

It is possible to define \textbf{right} $H$-module-algebras: it is an
algebra with a right $H$-module structure such that $(ab)\leftarrow
h=(a\leftarrow h_{1})(b\leftarrow h_{2})$ and $1\cdot h=\varepsilon(h)1$ for
all $h\in H$ and $a,b\in A$. 

\begin{exercise}
Let $H$ be a Hopf algebra.  Prove that $H^*$ is an left $H$-module-algebra
via  $\langle h\rightharpoonup f|x\rangle=\langle f|xh\rangle$ for all $f\in
H^*$, $h,x\in H$.  Similarly, prove that $H^*$ is a right $H$-module-algebra
via $\langle f\leftharpoonup h|x\rangle=\langle f|xh\rangle$.
\end{exercise}

\begin{exercise}
\label{exercise:adjoint}
\index{adjoint representation}
Let $H$ be a Hopf algebra. Define 
\begin{equation}
a\rightarrow x=a_{1}xS(a_{2})\label{eq:left_adjoint_action}
\end{equation}
for all $a,x\in H$. Prove that $(H,\rightarrow)$ is a left $H$-module-algebra.
The representation \ref{eq:left_adjoint_action} is called the \textbf{left
adjoint representation} of $H$. Similarly, prove that the \textbf{right
adjoint action}
\begin{equation}
x\leftarrow a=S(a_{1})xa_{2}\label{eq:right_adjoint_action}
\end{equation}
gives a right module-algebra over $H$.
\end{exercise}

Let $G$ be a group and $\mathbb{K}[G]$ be the corresponding Hopf algebra. Then
the right adjoint action is given by $a\rightarrow x=axa^{-1}$.

\begin{example}
Let $L$ be a Lie algebra and $U(L)$ be the enveloping algebra with
the canonical Hopf algebra structure. The right adjoint action
is given by \[
a\rightarrow x=ax-xa.
\]
\end{example}

\begin{exercise}
\label{exercise:left_smash}
\index{smash product!left}
Let $H$ be a bialgebra and let $(A,\rightarrow)$ be an left $H$-module-algebra.
There exists an algebra structure on $A\otimes H$ given by
\[
(a\otimes h)(b\otimes g)=a(h_{1}\rightarrow b)\otimes h_{2}g
\]
and unit $1\otimes1$. This algebra is called the \textbf{left smash product}
of $A$ and $H$.  Observe that the maps $A\to A\otimes H$, $a\mapsto a\otimes1$,
and $H\to A\otimes H$, $h\mapsto 1\otimes h$ are algebra embedings.
\end{exercise}

\begin{exercise}
\index{smash product!right}
Let $H$ be a Hopf algebra and $(A,\leftarrow)$ be an right $H$-module-algebra.
Prove that there exists an algebra structure on $H\otimes A$ given by 
\[
(h\otimes a)(g\otimes b)=hg_{1}\otimes(a\leftarrow g_{2})b
\]
and unit $1\otimes1$. This algebra is called the \textbf{right
smash product} of $H$ and $A$. 
\end{exercise}

%\section{Actions on coalgebras}

\begin{definition}
\label{def:module_coalgebra}
\index{module-coalgebra}
Let $H$ be a Hopf algebra. A left \textbf{$H$-module-coalgebra}
is a coalgebra $C$ with a left $H$-module structure such that 
\begin{align*}
(h\rightarrow c)_{1}\otimes(h\rightarrow c)_{2} & =(h_{1}\rightarrow c_{1})(h_{2}\rightarrow c_{2}),\\
\varepsilon(h\rightarrow c) & =\varepsilon(h)\varepsilon(c)
\end{align*}
for all $h\in H$ and $c\in C$. 
\end{definition}

A \textbf{right} $H$-module-coalgebra is a coalgebra $C$ with a right
$H$-module structure such that 
\begin{align*}
(c\leftarrow h)_{1}\otimes(c\leftarrow h)_{2} & =(c_{1}\leftarrow h_{1})(c_{2}\leftarrow h_{2})\\
\varepsilon(c\leftarrow h) & =\varepsilon(h)\varepsilon(c)
\end{align*}
for all $h\in H$, $c\in C$.

\begin{exercise}
\index{coadjoint action}
Let $H$ be a finite-dimensional Hopf algebra. Consider the actions
\[
(a\rightharpoonup f)(b)=f(ba),
\quad 
(f\leftharpoonup a)(b)=f(ab)
\]
for all $a,b\in H$, $f\in H^{*}$. The \textbf{left coadjoint action}
of $H$ on $H^{*}$ is 
\[
h\triangleright f=h_{1}\rightharpoonup f\leftharpoonup S^{-1}h_{2}=f(S^{-1}h_{2}?h_1),
\]
where $f(?)$ means the function $x\mapsto f(x)$. Prove that
$(H^*)^\mathrm{cop}$ is a left $H$-module-coalgebra via the left coadjoint
action. Similarly, the \textbf{right coadjoint action} of $H$ on $H^{*}$ is 
\[
f\triangleleft h=S^{-1}h_{1}\rightharpoonup f\leftharpoonup h_{2}=f(h_2?S^{-1}h_{1}).
\]
Prove that $H$ is a right $(H^*)^\mathrm{cop}$-module-coalgebra 
\end{exercise}

\begin{example}
Let $G$ be a finite group and $H=\mathbb{K}G$ be the group Hopf algebra. Then
$y\rightharpoonup e_{x}=e_{xy^{-1}}$ (resp. $e_{x}\leftharpoonup
y=e_{y^{-1}x}$) defines a left (resp. right) $H$-module structure over $H^{*}$.
The left coadjoint action of $H$ over $H^{*}$ is 
\[
y\triangleright e_{x}=y\rightharpoonup e_{x}\leftharpoonup y^{-1}=e_{xyx^{-1}}.
\]
\end{example}

\begin{exercise}
\index{regular action}
Let $H$ be a Hopf algebra and consider the \textbf{left regular action} of $H$ on itself:
$h\rightarrow g=gh$ for all
$h,g\in H$. Prove that $H$ is a left $H$-module-coalgebra. 
\end{exercise}

%\section{Coactions on algebras}

Recall that a \textbf{left $H$-comodule} is a pair $(V,\delta)$,
where $V$ is a vector space and $\delta:V\to H\otimes V$ is a linear
map such that 
\begin{align*}
(\id\otimes\delta)\delta & =(\Delta\otimes\id)\delta,\\
(\varepsilon\otimes\id)\delta & =\id.
\end{align*}
We write $\delta(v)=v_{-1}\otimes v_{0}$. Similarly, a \textbf{right
$H$-comodule} is a pair $(V,\delta)$, where $\delta:V\to V\otimes H$
is a linear map such that 
\begin{align*}
(\id\otimes\Delta)\delta & =(\delta\otimes\id)\delta,\\
(\id\otimes\varepsilon)\delta & =\id.
\end{align*}
In this case we write $\delta(v)=v_{0}\otimes v_{1}$.

\begin{definition}
\index{comodule-algebra}
Let $H$ be a Hopf algebra. An algebra $A$ is a said to be a left
\textbf{$H$-comodule-algebra} if $(A,\delta)$ is a left $H$-comodule and the
following properties are satisfied:
\begin{align*}
\delta(1_{A}) & =1_{H}\otimes1_{A},\\
\delta(ab) & =a_{-1}b_{-1}\otimes a_{0}b_{0}
\end{align*}
for all $a,b\in A$. (Here we write $\delta(a)=a_{-1}\otimes a_{0}\in H\otimes A$.)
\end{definition}

%\section{Coactions on coalgebras}

\begin{definition}
\index{comodule-coalgebra}
Let $H$ be a Hopf algebra. A coalgebra $C$ is said to be a left
\textbf{$H$-comodule-coalgebra} if $(C,\delta)$ is a left $H$-comodule and
the following properties are satisfied:
\begin{align*}
c_{-1}\varepsilon(c_{0}) & =\varepsilon(c)1,\\
(c_{1})_{-1}(c_{2})_{-1}\otimes(c_{1})_{0}\otimes(c_{2})_{0} & =c_{-1}\otimes(c_{0})_{1}\otimes(c_{0})_{2}
\end{align*}
for all $c\in C$.
\end{definition}

\begin{exercise}
\index{coadjoint coaction}
Let $H$ be a Hopf algebra. Consider the \textbf{left coadjoint coaction}
of $H$ on $H$: $\mathrm{coadj}(h)=h_{1}S(h_{3})\otimes h_{2}$ for $h\in H$. Prove that 
$H$ is a left $H$-comodule-coalgebra via the left coadjoint coaction.
\end{exercise}

\begin{exercise}
Let $H$ be a Hopf algebra, $C$ be a coalgebra and $f\in\hom(C,H)$
be a coalgebra map with convolution inverse $g$. Prove that $(C,\delta)$
is a left $H$-comodule coalgebra, where $\delta(c)=f(c_{1})g(c_{3})\otimes c_{2}$
for all $c\in C$. 
\end{exercise}

\begin{exercise}
\label{exercise:smash_coleft}
\index{smash coproduct!left}
Let $H$ be a Hopf algebra, and $(C,\delta)$ be a left $H$-comodule
coalgebra. Prove that $C\otimes H$ is a coalgebra with coproduct
\[
\Delta(c\otimes h)=\left(c_{1}\otimes c_{2,-1}h_{1}\right)\otimes\left(c_{2,0}\otimes h_{2}\right),
\]
and counit $\varepsilon(c\times h)=\varepsilon_{C}(c)\varepsilon_{H}(h)$ for
all $c\in C$, $h\in H$. This coalgebra structure on $C\otimes H$ is called the
\textbf{left smash coproduct}. Observe that the maps $C\otimes H\to C$,
$c\otimes h\mapsto c\varepsilon(h)$, and $C\otimes H\to H$, $c\otimes h\mapsto
\varepsilon(c)h$, are coalgebra surjections.
\end{exercise}

\index{smash coproduct!right}
Assume that $C$ is a right $H$-comodule coalgebra. The \textbf{right}
smash coproduct is then defined by 
\[
\Delta(h\otimes c)=h_{1}\otimes c_{1,0}\otimes h_{2}c_{1,1}\otimes c_{2}
\]
for all $h\in H$ and $c\in C$.

