\section{Yetter-Drinfeld modules}

\begin{definition}
\index{Yetter-Drinfeld module}
Let $H$ be a Hopf algebra. A \textbf{Yetter-Drifeld module} over $H$ is a
triple $(V,\rightarrow,\delta)$, where $(V,\rightarrow)$ is a left $H$-module,
$(V,\delta)$ is a left $H$-comodule, and such that 
\begin{equation}
\delta(h\rightarrow v)=h_{1}v_{-1}Sh_{3}\otimes h_{2}\rightarrow v_{0}\label{eq:YD}
\end{equation}
for all $h\in H$, $v\in V$. A \textbf{morphism} of Yetter-Drinfeld modules
is a morphism of left $H$-modules and left $H$-comodules. The category of Yetter-Drinfeld
modules will be denoted by $\ydH$.
\end{definition}

\begin{example}
Let $H$ be a Hopf algebra with the trivial action and coaction on itself:
$h\rightarrow x=\varepsilon(h)x$ and $\delta(h)=1\otimes h$ for all $h,x\in H$.
Then $(H,\rightarrow,\delta)$ is a Yetter-Drinfeld module over $H$.
\end{example}

\begin{example}
Let $H$ be a Hopf algebra. Then $(H,\mathrm{adj},\Delta)$ and
$(H,m,\mathrm{coadj})$ are Yetter-Drinfeld modules over $H$.
\end{example}

\begin{exercise}
\label{exercise:YD_condition}
Prove that the condition \eqref{eq:YD} is equivalent to
\begin{equation}
\label{eq:left_left_YD_equivalent}
h_{1}v_{-1}\otimes(h_{2}\rightarrow v_{0})=(h_{1}\rightarrow v)_{-1}h_{2}\otimes(h_{1}\rightarrow v)_{0}
\end{equation}
for all $h\in H$, $v\in V$.
\end{exercise}

\begin{exercise}
Let $G$ be a group, and $H$ be the group Hopf algebra of $G$. Assume that
$(V,\rightarrow)$ is a left $H$-module, and $(V,\delta)$ is a left
$H$-comodule. 
\begin{enumerate}
\item Prove that $V=\oplus_{g\in G}V_{g}$, where $V_{g}=\{v\in V\mid\delta(v)=g\otimes v\}$.  
\item Prove that the triple $(V,\rightarrow,\delta)$ is a Yetter-Drinfeld
module if and only if $h\rightarrow V_{g}\subseteq V_{hgh^{-1}}$ for all
$g,h\in H$.
\end{enumerate}
\end{exercise}

\begin{exercise}\label{exercise:YD_tensor}
Let $V$ and $W$ be two Yetter-Drinfeld modules over $H$. Then $V\otimes W$ is a
Yetter-Drinfeld over $H$, where 
\begin{align*}
h\rightarrow(v\otimes w) & =(h_{1}\rightarrow v)\otimes(h_{2}\rightarrow w),\\
\delta(v\otimes w) & =v_{-1}w_{-1}\otimes(v_{0}\otimes w_{0})
\end{align*}
for all $h\in H$, $v\in V$, $w\in W$.
\end{exercise}

Let $H$ be a Hopf algebra with invertible antipode. For any pair $V$ and $W$ of
Yetter-Drinfeld modules over $H$, we consider the map 
\begin{align*}
c_{V,W}:V\otimes W&\to W\otimes V\\
v\otimes w&\mapsto (v_{-1}\rightarrow w)\otimes v_{0}.
\end{align*}

\begin{lemma}
The map $c_{V,W}$ is an isomorphism in $\ydH$.
\end{lemma}

\begin{proof}
The map $c$ is invertible and the inverse is 
\begin{align*}
c_{V,W}^{-1}:W\otimes V & \to V\otimes W\\
w\otimes v & \mapsto v_{0}\otimes(S^{-1}(v_{-1})\to w)
\end{align*}
since
\begin{align*}
c_{V,W}^{-1}c_{V,W}(v\otimes w) & =c_{V,W}^{-1}((v_{-1}\to w)\otimes v_{0})\\
 & =v_{0,0}\otimes(S^{-1}(v_{0,-1})\to(v_{-1}\to w))\\
 & =v_{0,0}\otimes(S^{-1}(v_{0,-1})v_{-1}\to w)\\
 & =v_{0}\otimes(S^{-1}(v_{-1})v_{-2}\to w)\\
 & =v_{0}\otimes(\varepsilon(v_{-1})1\to w)\\
 & =v\otimes w,
\end{align*}
and similarly $c_{V,W}c_{V,W}^{-1}(w\otimes v)=w\otimes v$. 

Now we prove that $c_{V,W}$ is a morphism of $H$-modules: 
\begin{align*}
c_{V,W}(h\rightarrow (v\otimes w))&=c_{V,W}(h_1\rightarrow v\otimes h_2\rightarrow w)\\
&=(h_1\rightarrow v)_{-1}\rightarrow(h_2\rightarrow w)\otimes(h_1\rightarrow v)_0\\
&=(h_{11}v_{-1}Sh_{13})\rightarrow(h_2\rightarrow w)\otimes h_{12}\rightarrow v_0\\
&=(h_1v_{-1}(Sh_3)h_4)\rightarrow w\otimes h_2\rightarrow v_0\\
&=(h_1v_{-1})\rightarrow w\otimes h_2\rightarrow v_0\\
&=h_1\rightarrow(v_{-1}\rightarrow w)\otimes h_2\rightarrow v_0\\
&=h\rightarrow((v_{-1}\rightarrow w)\otimes v_0).
\end{align*}

To prove that $c_{V,W}$ is a morphism of comodules we need $(\id\otimes
c)\delta=\delta c$.  We compute:
\[
(\id\otimes c)\delta(v\otimes w)=v_{-1}w_{-1}\otimes (v_{0,-1}\rightarrow w_0)\otimes v_{0,0}.
\]
On the other hand,
\begin{align*}
\delta(c(v\otimes w)&=\delta(v_{-1}\rightarrow w\otimes v_0)\\
&=(v_{-1}\rightarrow w)_{-1}v_{0,-1}\otimes(v_{-1}\rightarrow w)_0\otimes v_{0,0}\\
&=(v_{-2}\rightarrow w)_{-1}v_{-1}\otimes (v_{-2}\rightarrow w)_0\otimes v_0\\
&=v_{-2,1}w_{-1}S(v_{-2,3})v_{-1}\otimes(v_{-2,2}\rightarrow w_0)\otimes v_0\\
&=v_{-4}w_{-1}S(v_{-2})v_{-1}\otimes (v_{-3}\rightarrow w_0)\otimes v_0\\
&=v_{-2}w_{-1}\otimes(v_{-1}\rightarrow w_0)\otimes v_0.
\end{align*}
This completes the proof.
\end{proof}

\begin{exercise}
\label{exercise:YD_hexagons}
Let $H$ be a Hopf algebra, and let $U$, $V$ and $W$
be three objects of $\ydH$. Prove that 
\begin{align}
c_{U\otimes V,W} & =(c_{U,W}\otimes\id_{V})(\id_{U}\otimes c_{V,W}),\label{eq:(cx1)(1xc)}\\
c_{U,V\otimes W} & =(\id_{V}\otimes c_{U,W})(c_{U,V}\otimes\id_{W}).\label{eq:(1xc)(cx1)}
\end{align}
\end{exercise}

\begin{exercise}
\label{exercise:YD_naturality}
Let $H$ be a Hopf algebra. Prove that 
\[
c_{V',W'}(f\otimes g)=(g\otimes f)c_{W,V}
\]
for all Yetter-Drinfeld modules morphisms $f:V\to V'$ and $g:W\to W'$. 
\end{exercise}

\begin{theorem}
\label{theorem:YD_braid_equation}
Let $H$ be a Hopf algebra with invertible antipode, and let $U,V,W$
be Yetter-Drinfeld modules over $H$. Then 
%\begin{align*}
%c_{V,W}:V\otimes W&\to W\otimes V\\
%v\otimes w&\mapsto (v_{-1}\rightarrow w)\otimes v_{0},
%\end{align*}
%is an isomorphism in $_{H}^{H}\mathcal{YD}$ and it is a solution of the braid
%equation:
\begin{align*}
(c_{V,W}\otimes\textrm{id}_{U})(\textrm{id}_{V}&\otimes c_{U,W})(c_{U,V}\otimes\textrm{id}_{W})\\
&=(\textrm{id}_{W}\otimes c_{U,V})(c_{U,W}\otimes\textrm{id}_{V})(\textrm{id}_{U}\otimes c_{V,W}).
\end{align*}
\end{theorem}

The proof follows from Exercise~\ref{exercise:YD_naturality} with
$f=c_{U,V}\otimes\id_W$ and $g=\id_W$ and Exercise
\ref{exercise:YD_hexagons}.

\begin{exercise}
Prove Theorem \ref{theorem:YD_braid_equation} without using Exercises
\ref{exercise:YD_naturality} and \ref{exercise:YD_hexagons}.
\end{exercise}

%It remains
%to prove that $c$ is natural, i.e., 
%\[
%(g\otimes f)c_{V,W}=c_{V',W'}(f\otimes g).
%\]
%So let $V$, $V'$, $W$ and $W'$ be objects of $_{H}^{H}\mathcal{YD}$,
%and $f\in\hom(V,V')$, $g\in\hom(W,W')$. We compute 
%\begin{align*}
%(g\otimes f)c_{V,W}(v\otimes w) & =(g\otimes f)(v_{-1}\to w\otimes v_{0})\\
% & =g(v_{-1}\to w)\otimes f(v_{0})\\
% & =v_{-1}\to g(w)\otimes f(v_{0})
%\end{align*}
%and
%\begin{align*}
%c_{V',W'}(f\otimes g)(v\otimes w) & =f(v)\otimes g(w)\\
% & =f(v)_{-1}\to g(w)\otimes f(v)_{0}\\
% & =v_{-1}\to g(w)\otimes f(v_{0})
%\end{align*}
%(here we use that $f$ is morphism of $H$-comodules). Hence the claim
%holds.

%\subsection{The category $_H\mathcal{YD}^H$}

We will also work with the following variation of what a Yetter-Drinfeld module
is: An object $V$ in the category 
$\prescript{}{H}{\mathcal{YD}^H}$
$_{H}\mathcal{YD}^{H}$ 
is a triple
$(V,\rightarrow,\delta)$, where $(V,\rightarrow)$ is a left $H$-module,
$(V,\delta)$ is a right $H$-comodule, such that
\[
h_{1}\rightarrow v_{0}\otimes h_{2}v_{1}=(h_{2}\rightarrow v)_{0}\otimes(h_{2}\rightarrow v)_{1}h_{1},
\]
or equivalently
\[
\delta(h\rightarrow v)=h_{2}\rightarrow v_{0}\otimes h_{3}v_{1}S^{-1}h_{1},
\]
for all $v\in V$, $h\in H$. 
%\end{rem}

%\begin{exercise}
%Let $H$ be a Hopf algebra with bijective antipode. Prove that the categories
%$_{H}^{H}\mathcal{YD}$ and $_{H}\mathcal{YD}^{H}$ are equivalent.
%\end{exercise}
%
%\begin{solution}
%Let $(V,\rightarrow,\delta)$ be an object of $_{H}^{H}\mathcal{YD}$, where we
%write $\delta(v)=v_{-1}\otimes v_{0}$. Let $\rho:V\to V\otimes H$ be the linear
%map defined by $\rho(v)=Sv_{1}\otimes v_{0}$. Then $(V,\rightarrow,\rho)$ is an
%object of $_{H}\mathcal{YD}^{H}$. The converse is similar. 
%\end{solution}

\begin{exercise}
Let $H$ be a finite-dimensional Hopf algebra with bijective antipode.  Assume that
$(V,\rightarrow,\delta_R)$ is an object of $_{H}\mathcal{YD}^{H}$ and define
\[
\delta_L(v)=S(v_1)\otimes v_0
\]
for all $v\in V$.  Prove that
$(V,\rightarrow,\delta_L)$ is an object of $\ydH$. 
Conversely, if $(V,\rightarrow,\delta_L)$ is an object of $\ydH$,
define \[
\delta_R(v)=v_0\otimes S^{-1}v_{-1}
\]
for all $v\in V$. Prove that
$(V,\rightarrow,\delta_R)$ is an object of $_H\mathcal{YD}^H$.
\end{exercise}

\subsection{Yetter-Drinfeld modules and the Drinfeld double}

\begin{exercise}
Let $H$ be a finite-dimensional Hopf algebra. Assume that $\{h_i\}$ is a basis
of $H$, and let $\{h^i\}$ be its dual basis.  Prove that the element
\[
\sum h^i\otimes h_i
\]
does not depend on the pair of dual basis $\{h_i\}$ and $\{h^i\}$.
\end{exercise}

\begin{lemma}
\label{lem:DH_compatibility}
Let $H$ be a finite-dimensional Hopf algebra. Then  $V$ is a left
$\mathcal{D}(H)$-module if and only if $V$ is a left $H$-module, a left
$H^{*}$-module and 
\begin{eqnarray}
h\cdot(f\cdot v) & = & f(S^{-1}(h_{3})?h_{1})\cdot(h_{2}\cdot v)\label{eq:compatibility_D(H)}
\end{eqnarray}
for all $h\in H$, $f\in H^{*}$.
\end{lemma}

\begin{proof}
We compute 
\begin{align*}
(1\otimes h)\cdot((f\otimes1)\cdot v) & =((1\otimes h)(f\otimes1))\cdot v\\
 & =(f(S^{-1}(h_{3})?h_{1})\otimes h_{2})\cdot v\\
 & =(f(S^{-1}(h_{3})?h_{1})\otimes1)(1\otimes h_{2}))\cdot v\\
 & =f(S^{-1}(h_{3})?h_{1})\cdot(h_{2}\cdot v).
\end{align*}
and the claim follows. 
\end{proof}

\begin{lemma}
\label{lem:DH_to_YD}
Let $H$ be a finite-dimensional Hopf algebra and assume that $\{h_i\}$ is a basis
of $H$, and let $\{h^i\}$ be its dual basis.  
Let $(V,\cdot)$ be a left $\mathcal{D}(H)$-module. For any $v\in V$ define 
\[
\delta(v)=\sum h^i\cdot v\otimes h_i.
\]
Then the triple $(V,\cdot,\delta)$ is an object of $_H\mathcal{YD}^H$.
\end{lemma}

\begin{proof}
We prove the compatibility condition
\begin{equation}
\label{eq:DH_to_YD}
\sum h^i\cdot (v\cdot v)\otimes h_i=\sum x_2\cdot(h^i\cdot v)\otimes x_3h_iS^{-1}x_1
\end{equation}
for all $x\in H$, $v\in V$. Let $f\in H^*$ and apply $(\id\otimes f)$ to the
left hand side of \eqref{eq:DH_to_YD} to obtain
\[
\sum h^i\cdot (x\cdot v)f(h_i)=f\cdot (x\cdot v).
\]
On the other hand, applying $(\id\otimes f)$ to the right hand side of
\eqref{eq:DH_to_YD} we obtain
\begin{align*}
%(\id\otimes f)&\left(\sum x_2\cdot (h^i\cdot v)\otimes x_3h_1S^{-1}x_1\right)\\
\sum x_2\cdot(h^i\cdot v) f(x_3h_iS^{-1}x_1)&=
x_2\cdot\left( f(x_3?S^{-1}x_1)\cdot v\right)\\
&=f(x_3S^{-1}x_{23}?x_{21}S^{-1}x_1)\cdot (x_{22}\cdot v)\\
&=f(x_5S^{-1}x_4?x_2S^{-1}x_1)\cdot(x_3\cdot v)\\
&=f\cdot (x\cdot v)
\end{align*}
and the claim follows.
\end{proof}

\begin{lemma}
\label{lem:YD_to_DH}
Let $H$ be a finite-dimensional Hopf algebra. Let 
$(V,\cdot,\delta)$ be an object of $_H\mathcal{YD}^H$. Then $V$ is a left
$\mathcal{D}(H)$-module via 
\[
(f\otimes h)\cdot v=\langle f\mid (h\cdot v)_1\rangle (h\cdot v)_0
\]
for all $f\in H^*$, $h\in H$ and $v\in V$.
\end{lemma}

\begin{proof}
By Lemma \ref{lem:DH_compatibility}, we need prove that
\[
h \cdot(f\cdot v)=\langle f\mid v_1\rangle(h\cdot v_0)
\]
for all $f\in H^*$, $h\in H$, $v\in V$. We compute:
\begin{align*}
f(S^{-1}h_3?h_1)\cdot(h_2\cdot v)&=\langle f\mid S^{-1}h_3(h_2\cdot v)_1h_1\rangle(h_2\cdot v)_0\\
&=\langle f\mid S^{-1}h_3(h_{23}v_1S^{-1}h_{21})h_1\rangle(h_{22}\cdot v_0)\\
&=\langle f\mid S^{-1}h_5h_4v_1S^{-1}h_2h_1\rangle h_3\cdot v_0\\
&=\langle f\mid v_1\rangle (h\cdot v_0)
\end{align*}
and the claim follows.
\end{proof}

\begin{theorem}
The categories $_H\mathcal{YD}^H$ and $_{\mathcal{D}(H)}\mathcal{M}$ are
equivalent.
\end{theorem}

\begin{proof}
It follows from Lemmas \ref{lem:DH_to_YD} and \ref{lem:YD_to_DH}.
\end{proof}

