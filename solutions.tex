\section{Some solutions}

%\subsection*{Quasitriangular Hopf algebras}

\begin{sol}{exercise:QT_hexagons}
We prove \eqref{eq:(Rx1)(1xR)}. A straightforward computation shows
that 
\[
R_{U\otimes V,W}(u\otimes v\otimes w)=\sum b_{i}\cdot w\otimes a_{i1}\cdot u\otimes a_{i2}\cdot v.
\]
On the other hand,
\begin{align*}
(R_{U,W}\otimes\id_{V}) & (\id_{U}\otimes R_{V,W})(u\otimes v\otimes w)\\
 & =\sum(R_{U,W}\otimes\id_{V})(u\otimes b_{i}\cdot w\otimes a_{i}\cdot v)\\
 & =\sum(b_{j}b_{i})\cdot w\otimes a_{j}\cdot w\otimes a_{i}\cdot w
\end{align*}
and the claim follows from Equation \eqref{QT:2}. The proof for \eqref{eq:(1xR)(Rx1)}
is similar. 
\end{sol}

\begin{sol}{exercise:VW=WV}
Define $\phi:V\otimes W\to W\otimes V$ by $v\otimes w\mapsto R^{-1}\cdot (w\otimes
v)$. Then $\phi$ is an isomorphism of left $H$-modules:
\begin{align*}
\phi(h\cdot (v\otimes w))&=R^{-1}(h_2\cdot w\otimes h_1\cdot v)\\
&=R^{-1}\tau\Delta(h)(w\otimes v)=\Delta(h)R^{-1}(w\otimes v)=h\cdot \phi(v\otimes w).
\end{align*}
\end{sol}

%\subsection*{Actions and coactions}

\begin{sol}{exercise:adjoint}
First we prove that $(H,\rightarrow)$ is a left $H$-module.  We
compute 
\begin{align*} 
b\rightarrow(a\rightarrow x) & =b\rightarrow(a_{1}xS(a_{2}))\\
 & =b_{1}a_{1}xS(a_{2})S(b_{2})\\
 & =(ba)_{1}xS\left((ba)_{2}\right)\\
 & =(ba)\rightarrow x.
\end{align*}
Then $(H,\rightarrow)$ is a left $H$-module, since it is trivial
to prove that $1\rightarrow x=x$. To prove that $(H,\rightarrow)$
is a left module-algebra over $H$ we compute:
\[
a\rightarrow=a_{1}1S(a_{2})=\varepsilon(a)1,
\]
and
\begin{align*}
(a_{1}\rightarrow x)(a_{2}\rightarrow y) & =(a_{1,1}xS(a_{1,2}))(a_{2,1}yS(a_{2,2}))\\
 & =a_{1}x\varepsilon(a_{2})yS(a_{3})\\
 & =a_{1}xyS(a_{2})\\
 & =a\rightarrow(xy).
\end{align*}
The proof for the right adjoint action is similar. 
\end{sol}

\begin{sol}{exercise:left_smash}
We first prove that $1\otimes1$ is the unit: 
\begin{align*}
(1\otimes1)(a\otimes h) & =1(1\rightarrow a)\otimes1h=1a\otimes h=a\otimes h,\\
(a\otimes h)(1\otimes1) & =a(h_{1}\rightarrow1)\otimes h_{2}1=a(\varepsilon(h_{1})1)\otimes h_{2}=a\otimes h.
\end{align*}
Now we prove the associativity. A direct computation shows that 
\begin{align*}
\left((a\otimes h)(b\otimes g)\right)(c\otimes k) & =(a(h_{1}\rightarrow b)\otimes h_{2}g)(c\otimes k)\\
 & =(a(h_{1}\rightarrow b))((h_{2}g)_{1}\rightarrow c)\otimes(h_{2}g)_{2}k\\
 & =a(h_{1}\rightarrow b)(h_{2}g_{1}\rightarrow c)\otimes(h_{3}g_{2})k.
\end{align*}
On the other hand, since $A$ is an $H$-module-algebra,
\begin{align*}
(a\otimes h)\left((b\otimes g)(c\otimes k)\right) & =(a\otimes h)(b(g_{1}\rightarrow c)\otimes g_{2}k)\\
 & =a(h_{1}\rightarrow(b(g_{1}\rightarrow c)))\otimes h_{2}(g_{2}k)\\
 & =a(h_{1}\rightarrow b)(h_{2}\rightarrow(g_{1}\rightarrow c))\otimes h_{3}(g_{2}k).
\end{align*}
\end{sol}

\begin{sol}{exercise:smash_coleft}
We first prove that $\varepsilon$ is the counit:
\begin{align*}
(\varepsilon\otimes\textrm{id})\Delta(c\otimes h) & =(\varepsilon\otimes\textrm{id})(c_{1}\otimes c_{2,-1}h_{1}\otimes c_{2,0}\otimes h_{2})\\
 & =\varepsilon(c_{1}\otimes c_{2,-1}h_{1})c_{2,0}\times h_{2}\\
 & =\varepsilon_{C}(c_{1})\varepsilon_{H}(c_{2,-1}h_{1})c_{2,0}\otimes h_{2}\\
 & =\varepsilon_{C}(c_{1})\varepsilon_{H}(h_{1})(c_{2,-1})c_{2,0}\otimes\varepsilon_{H}(h_{1})h_{2}\\
 & =c\otimes h,
\end{align*}
where the last equality holds since $(\varepsilon_{H}\otimes\textrm{id})\delta=\textrm{id}$
and hence 
\[
c=(\varepsilon_{H}\otimes\textrm{id})\delta(c)=(\varepsilon_{H}\otimes\textrm{id})\delta(\varepsilon_{C}(c_{1})c_{2})=\varepsilon_{C}(c_{1})\varepsilon_{H}(c_{2,-1})c_{2,0}.
\]
Similarly we obtain that $\textrm{(id}\otimes\varepsilon)\Delta=\textrm{id}$.
Now we prove the coassociativity: 
\begin{align*}
(\Delta\otimes\textrm{id})\Delta(c\otimes h) & =(\Delta\otimes\textrm{id})((c_{1}\otimes c_{2,-1}h_{1})\otimes(c_{2,0}\otimes h_{2}))\\
 & =\Delta(c_{1}\otimes c_{2,-1}h_{1})\otimes(c_{2,0}\otimes h_{2})\\
 & =c_{1,1}\otimes c_{1,2,-1}(c_{2,-1}h_{1})_{1}\otimes c_{1,2,0}\otimes(c_{2,-1}h_{1})_{2}\otimes c_{2,0}\otimes h_{2}\\
 & =c_{1}\otimes c_{2,-1}(c_{3,-1}h_{1})_{1}\otimes c_{2,0}\otimes(c_{3,-1}h_{1})_{2}\otimes c_{3,0}\otimes h_{2}\\
 & =c_{1}\otimes c_{2,-1}c_{3,-1,1}h_{1}\otimes c_{2,0}\otimes c_{3,-1,2}h_{2}\otimes c_{3,0}\otimes h_{3}\\
 & =c_{1}\otimes c_{2,-1}c_{3,-1}h_{1}\otimes c_{2,0}\otimes c_{3,0,-1}h_{2}\otimes c_{3,0,0}\otimes h_{3}\\
 & =c_{1}\otimes c_{2,-1}c_{3,-2}h_{1}\otimes c_{2,0}\otimes c_{3,-1}h_{2}\otimes c_{3,0}\otimes h_{3},
\end{align*}
where we have used that $C$ is a left $H$-comodule-coalgebra: 
\[
c_{-1,1}\otimes c_{-1,2}\otimes c_{0}=c_{-1}\otimes c_{0,-1}\otimes c_{0,0}=c_{-2}\otimes c_{-1}\otimes c_{0}\in H\otimes H\otimes C.
\]
On the other hand, 
\begin{align*}
(\textrm{id}\otimes\Delta)\Delta(c\otimes h) & =(\textrm{id}\otimes\Delta)((c_{1}\otimes c_{2,-1}h_{1})\otimes(c_{2,0}\otimes h_{2}))\\
 & =c_{1}\otimes c_{2,-1}h_{1}\otimes\Delta(c_{2,0}\otimes h_{2})\\
 & =c_{1}\otimes c_{2,-1}h_{1}\otimes c_{2,0,1}\otimes c_{2,0,2,-1}h_{2,1}\otimes c_{2,0,2,0}\otimes h_{2,2}\\
 & =c_{1}\otimes c_{2,-1}h_{1}\otimes c_{2,0,1}\otimes c_{2,0,2,-1}h_{2}\otimes c_{2,0,2,0}\otimes h_{3}\\
 & =c_{1}\otimes c_{2,-1}c_{3,-1}h_{1}\otimes c_{2,0}\otimes c_{3,0,-1}h_{2}\otimes c_{3,0,0}\otimes h_{3}\\
 & =c_{1}\otimes c_{2,-1}c_{3,-2}h_{1}\otimes c_{2,0}\otimes c_{3,-1}h_{2}\otimes c_{3,0}\otimes h_{3},
\end{align*}
where we have used that $c_{-1}\otimes c_{0,1}\otimes c_{0,2}=c_{1,-1}c_{2,-1}\otimes c_{1,0}\otimes c_{2,0}$
since $C$ is a left $H$-comodule-coalgebra.
\end{sol}

\begin{sol}{exercise:YD_condition}
Assume that \eqref{eq:YD} holds. Then 
\[
\delta(h_{1}\to v)=(h_{1}\to v)_{-1}\otimes(h_{1}\to v)_{0}=h_{1,1}v_{-1}Sh_{1,3}\otimes h_{1,2}\to v_{0}.
\]
Hence 
\[
(h_{1}\to v)_{-1}h_{2}\otimes(h_{1}\to v)_{0}=h_{1,1}v_{-1}Sh_{1,3}h_{2}\otimes h_{1,2}\to v_{0}=h_{1}v_{-1}\otimes h_{2}\to v_{0}.
\]
Conversely, assume that \eqref{eq:left_left_YD_equivalent} holds. Then
\begin{align*}
	(m\otimes&\id)(h_{11}v_{-1}\otimes Sh_2\otimes(h_{12}\rightarrow v_0) )\\  
	&=(m\otimes\id)\left( (h_{11}\rightarrow v)_{-1}h_{12}\otimes Sh_2\otimes (h_{11}\rightarrow v)_0 \right)\\
	&=(h_1\rightarrow v)_{-1}h_2Sh_3\otimes (h_1\rightarrow v)_0\\
	&=(h\rightarrow v)_{-1}\otimes (h\rightarrow v)_0.
\end{align*}
\end{sol}

\begin{sol}{exercise:YD_tensor}
To prove the compatibility condition \eqref{eq:YD} we compute
\begin{align*}
\delta(h\rightarrow(v\otimes w)) & =\delta(h_{1}\rightarrow v\otimes h_{2}\rightarrow w)\\
 & =(h_{1}\rightarrow v)_{-1}(h_{2}\rightarrow w)_{-1}\otimes(h_{1}\rightarrow v)_{0}\otimes(h_{2}\rightarrow w)_{0}\\
 & =(h_{1}v_{-1}(Sh_{3})h_{4}w_{-1}Sh_{6}\otimes(h_{2}\rightarrow v_{0})\otimes(h_{5}\rightarrow w_{0})\\
 & =h_{1}v_{-1}w_{-1}Sh_{4}\otimes(h_{2}\rightarrow v_{0})\otimes(h_{3}\rightarrow w_{0})\\
 & =h_{1}v_{-1}w_{-1}Sh_{3}\otimes h_{2}\rightarrow(v_{0}\otimes w_{0})\\
 & =h_{1}(v\otimes w)_{-1}Sh_{3}\otimes h_{2}\rightarrow(v\otimes w)_{0}.
\end{align*}
\end{sol}


\begin{sol}{exercise:triangular}
Assume first that $H$ es triangular. Then $\tau(R)=R$ and hence
$c_{V,W}c_{W,V}=\id_{V\otimes W}$. Conversely, using 
\eqref{eq:QT_auxiliar} we obtain 
\[
1\otimes1=c_{H,H}(c_{H,H}(1\otimes1))=c_{H,H}(\tau(R))=\tau(R)R.
\]
\end{sol}